% Autor: Alfredo Sánchez Alberca (asalber@ceu.es)

\newproblem{derimpn-1}{gen}{}
%STATEMENT
{Suppose the equation $F(x,y,z)=0$ defines the variable $z$ as a function of the variables $x$ and $y$ ($z=f(x,y)$).
Show that
\[
\frac{\partial z}{\partial x} = \frac{-\dfrac{\partial F}{\partial x}}{\dfrac{\partial F}{\partial z}}
\quad \mbox{and} \quad
\frac{\partial z}{\partial y} = \frac{-\dfrac{\partial F}{\partial y}}{\dfrac{\partial F}{\partial z}}.
\]
Use these identities to compute $\dfrac{\partial f}{\partial x}(2,1)$, when we have $x^2yz=4$ and $f(2,1)=1$.
}
%SOLUTION
{$\frac{\partial f}{\partial x}(2,1)=-1.$
}
%RESOLUTION
{
}

\newproblem{derimpn-2}{gen}{*}
%STATEMENT
{The equation
\[
x\log y+\frac{2e^{y^2+z}}{x}-\frac{x}{z^2} = -1
\]
defines $z$ as a function of $x$ and $y$ in a neighborhoud of the point $(2,1,-1)$.
Compute the gradient of $z$ at that point, and explain its meaning.
}
%SOLUTION
{$\nabla z(2,1,-1) = (-1/2,4/3)$.
}
%RESOLUTION
{
}


\newproblem*{derimpn-3}{gen}{*}
%STATEMENT
{Consider two surfaces, $S$ and $M$, where $S$ is given by the equation $xy+8z=0$; and $M$ is the ellipsoid with equation $x^2+2y^2+4z^2=7$.
Find the equation of the plane tangent to $S$, and parallel to the tangent plane to $M$ at the point $P=(1,1,1)$.
}
%SOLUTION
{
}
%RESOLUTION
{
}
