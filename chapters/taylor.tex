% Autor: Alfredo Sánchez Alberca (asalber@ceu.es)

\newproblem*{tay-1}{gen}{}
%ENUNCIADO
{Consider the function $f(x)=\sqrt{x+1}$.
\begin{enumerate}
\item  Compute the Taylor polynomial of $f$, of degree $4$, centered at $x=0$.
\item Obtain two approximate values of $\sqrt{1.02}$ by means of degrees 2 and 4 Taylor polynomials.
\end{enumerate}
\end{enumerate}
}


\newproblem{tay-2}{gen}{}
%ENUNCIADO
{Consider the sine function $f(x) = \sin x$.
\begin{enumerate}
\item Compute the third degree Taylor polynomial centered at the point $x=\pi/6$.
Use this polynomial to approximate $\sin (1/2)$. %Provide an upper bound for the error in your approximation.
\item Give an approximate value of $\sin (1/2)$ using a fifth degree Taylor polynomial centered at the point $x=0$. %Give an upper bound for the error in your approximation.
\end{enumerate}
}
%SOLUCIÓN
{
\begin{enumerate}
\item $P^3_{f,\pi/6}(x) = \frac{1}{2}+\frac{\sqrt{3}}{2}(x-\pi/6)-\frac{1}{4}(x-\pi/6)^2-\frac{\sqrt{3}}{12}(x-\pi/6)^3$.\\
$\sen 1/2 \approx P^3_{f,\pi/6}(1/2) = 0.4794255322$.\\
%$|R^3_{f,\pi/6}(1/2)|\leq 6.46\cdot 10^{-9}$.
\item $P^5_{f,0}(x) = x -\frac{1}{6} x^3 + \frac{1}{120}x^5$.\\
$\sen 1/2 \approx P^5_{f,0}(1/2) = 0.4794270833$.\\
%$|R^5_{f,0}(1/2)|\leq 2.170\cdot 10^{-5}$.
\end{enumerate}
}
%RESOLUCIÓN
{
}


\newproblem{tay-3}{gen}{}
%ENUNCIADO
{Compute the second degree Taylor polynomial of the funtion $f(x)=\sqrt[3]{x}$ in a neighborhood of the point $x=1$.
}
%SOLUCIÓN
{$P^2_{f,1}(x) = 1+\frac{1}{3}(x-1)-\frac{2}{18}(x-1)^2$.
}
%RESOLUCIÓN
{
}


\newproblem{tay-4}{gen}{}
%ENUNCIADO
{Compute the third degree Maclaurin polynomial for the function $f(x)=\arcsin x$.
}
%SOLUCIÓN
{$P^3_{f,0}(x) = x+\frac{1}{6}x^3$.
}
%RESOLUCIÓN
{
}


\newproblem*{tay-5}{gen}{}
%ENUNCIADO
{Calcular $\cos 1$ con un error menor de $10^{-7}$ usando aproximaciones de Taylor.
}


\newproblem{tay-6}{gen}{*}
%ENUNCIADO
{Dadas las funciones
$f(x)=e^x$ y $g(x)=\cos x$, se pide:
\begin{enumerate}
   \item  Calcular los polinomios de Maclaurin de segundo grado para $f$
   y $g$.

   \item  Utilizar los polinomios anteriores para calcular
   \[ \lim_{x\rightarrow 0}\frac{e^x-\cos x}{x}.\]
\end{enumerate}
}
%SOLUCIÓN
{\begin{enumerate}
\item $P^2_{f,0}(x) = 1+x+\frac{1}{2}x^2$ y $P^2_{g,0}(x) = 1-\frac{1}{2}x^2$.
\item $\lim_{x\rightarrow 0}\frac{e^x-\cos x}{x} = \lim_{x\rightarrow 0}\frac{x+x^2}{x} = 1$.
\end{enumerate}
}
%RESOLUCIÓN
{
}


\newproblem{tay-7}{far}{*}
%ENUNCIADO
{The function $C(t)$ measures the concentration (in mg/dl) of a drug in the bloodstream as function of time (in hours):
\[
C(t) = \frac{1}{{1 + e^{-2t}}}
\]
\begin{enumerate}
\item Compute the third degree Maclaurin polynomial for the function.

\item Use the previous polynomial to compute approximately the concentration of drug in the bloodstream after 15 minutes.
\end{enumerate}
}
%SOLUCIÓN
{
\begin{enumerate}
\item $P_{C,0}^3(t)=\frac{1}{2}+\frac{1}{2}t+0\frac{t^2}{2!}-1\frac{t^3}{3!}=\frac{1}{2}+\frac{1}{2}t-\frac{1}{6}t^3$.
\item $P_{C,0}^3(0.25)= 0.6223958333 \mbox{ mg/dl}$.
\end{enumerate}
}
%RESOLUCIÓN
{\begin{enumerate}
\item
La fórmula del polinomio de Maclaurin de orden 3 para la función $C(t)$ es:
\begin{equation}
\label{e:Maclaurin}
P_{C,0}^3(t)=C(0)+\frac{dC}{dt}(0)t+\frac{d^2C}{dt^2}(0)\frac{t^2}{2!}+\frac{d^3C}{dt^3}(0)\frac{t^3}{3!}
\end{equation}
Necesitamos calcular las tres primeras derivadas:
\begin{align*}
\frac{dC}{dt} &= \frac{2e^{-2t}}{(1+e^{-2t})^2},\\
\frac{d^2C}{dt^2} &=
\frac{\frac{d}{dt}(2e^{-2t})(1+e^{-2t})^2-2e^{-2t}\frac{d}{dt}(1+e^{-2t})^2}{(1+e^{-2t})^4} =\\
&= \frac{-4e^{-2t}(1+e^{-2t})^2- 2e^{-2t}2(1+e^{-2t})(-2e^{-2t})}{(1+e^{-2t})^4}
= \frac{-4e^{-2t}+4e^{-4t}}{(1+e^{-2t})^3},\\
\frac{d^3C}{dt^3}
&=
\frac{\frac{d}{dt}(-4e^{-2t}1+4e^{-4t})(1+e^{-2t})^3-(-4e^{-2t}+4e^{-4t})\frac{d}{dt}(1+e^{-2t})^3}{(1+e^{-2t})^6}=\\
&=
\frac{(8e^{-2t}-16e^{-4t})(1+e^{-2t})^3-(-4e^{-2t}+4e^{-4t})3(1+e^{-2t})^2(-2e^{-2t})}{(1+e^{-2t})^6}=\\
&=
\frac{(8e^{-2t}-16e^{-4t})(1+e^{-2t})-(-4e^{-2t}+4e^{-4t})(-6e^{-2t})}{(1+e^{-2t})^4}=\\
&=
\frac{(8e^{-2t}-8e^{-4t}-16e^{-6t})-(24e^{-4t}-24e^{-6t})}{(1+e^{-2t})^4}=\\
&=
\frac{8e^{-2t}-32e^{-4t}+8e^{-6t}}{(1+e^{-2t})^4}.
\end{align*}
Sustituyendo para $t=0$ tenemos:
\begin{align*}
C(0)&= \frac{1}{1+e^{-2\cdot 0}}=\frac{1}{2},\\
\frac{dC}{dt}(0) &= \frac{2e^{-2\cdot 0}}{(1+e^{-2\cdot 0})^2} = \frac{2}{2^2}=\frac{1}{2},\\
\frac{d^2C}{dt^2}(0) &= \frac{-4e^{-2\cdot 0}+4e^{-4\cdot 0}}{(1+e^{-2\cdot
0})^3} = \frac{-4+4}{2^3}= 0,\\
\frac{d^3C}{dt^3}(0)&=\frac{(8e^{-2\cdot 0}-32e^{-4\cdot 0}+8e^{-6\cdot 0})}{(1+e^{-2\cdot 0})^4}=\frac{8-32+8}{16}=-1.
\end{align*}

Y por último, sustituyendo en la ecuación \ref{e:Maclaurin} llegamos al polinomio
\[
P_{C,0}^3(t)=\frac{1}{2}+\frac{1}{2}t+0\frac{t^2}{2!}-1\frac{t^3}{3!}=\frac{1}{2}+\frac{1}{2}t-\frac{1}{6}t^3.
\]

\item La concentración del fármaco transcurridos 15 minutos ($0.25$ horas) es aproximadamente
\[
C(0.25)\approx P_{C,0}^3(0.25)= \frac{1}{2}+\frac{1}{2}0.25-\frac{1}{6}0.25^3= 0.6223958333 \mbox{ mg/dl}.
\]
\end{enumerate}
}


\newproblem{tay-8}{gen}{*}
%ENUNCIADO
{Obtener el desarrollo en serie de Taylor hasta orden tres en el punto $1$ de la función
\[f(x) = \ln\sqrt{\dfrac{x^2+1}{2}}.\]
Utilizar el polinomio obtenido calcular una aproximación de $\ln\sqrt{1.22}$ y acotar el error cometido.
}
%SOLUCIÓN
{$P^3_{f,1}(x) = \frac{1}{2}(x-1)-\frac{1}{12}(x-1)^3$.\\
$\ln\sqrt{1.22} \approx P^3_{f,1}(1.2) = 0.0993333$.\\
$|R^3_{f,1}(1.2)|\leq 1.25\cdot 10^{-4}$.

}
%RESOLUCIÓN
{
}


\newproblem{tay-9}{gen}{*}
%ENUNCIADO
{Dada la función $f(x)=\arctg(x/2)$ se pide:
\begin{enumerate}
\item Calcular el polinomio de Maclaurin de orden 3.
\item Utilizar el polinomio anterior para aproximar $\arctg 0.05$.
\item Dar una cota del error cometido.
\end{enumerate}
}
%SOLUCIÓN
{\begin{enumerate}
\item $P^3_{f,0}(x) =  \frac{1}{2}x-\frac{1}{24}x^3$.
\item $\arctan 0.05 \approx P^3_{f,0}(0.1) = 0.0499583333$.
\item $|R^3_{f,0}(0.1)|\leq 3.08\cdot 10^{-7}$.
\end{enumerate}
}
%RESOLUCIÓN
{
}


\newproblem{tay-10}{gen}{*}
%ENUNCIADO
{Dada la función $f(x)=\dfrac{1}{1-x}$ con $x\neq 1$, se pide:
\begin{enumerate}
\item Calcular el polinomio de Maclaurin de $f$ de grado 4.
\item Calcular el polinomio de Maclaurin de $f$ de grado $n$.
\item Calcular el resto de Lagrange para el polinomio de grado $n$ en el punto $x=0.03$.
\item ¿Hasta qué grado tendríamos que llegar para conseguir una aproximación de $f(0.03)$ con un error menor de $10^{-10}$?
\end{enumerate}
}
%SOLUCIÓN
{\begin{enumerate}
\item $P^4_{f,0}(x) = 1 + x + x^2 + x^3 + x^4$.
\item $P^n_{f,0}(x) = 1 + x + x^2 + \cdots + x^n$.
\item $R^n_{f,0}(0.03) = \frac{0.03^{n+1}}{(1-t)^{n+2}} \ t\in(0,0.03)$.
\item $|R^n_{f,0}(0.03)| \leq \frac{0.03^{n+1}}{(0.97)^{n+2}}\leq 10^{-10}$ para $n\geq 6$.
\end{enumerate}
}
%RESOLUCIÓN
{
}


\newproblem{tay-11}{gen}{*}
%ENUNCIADO
{Dada la función $f(x)=\tg(x/2)$, se pide:
\begin{enumerate}
\item Aproximar $\tg(0.1)$ mediante un polinomio de Taylor de grado 3 para la función $f$.
\item Dar una cota del error cometido.
\end{enumerate}
}
%SOLUCIÓN
{
\begin{enumerate}
\item $T_3 (f,0)(0.2) = 0.1003333$.
\item ${\rm Error} \le 0.00033$.
\end{enumerate}
}
%RESOLUCIÓN
{\begin{enumerate}
\item Sabemos que el desarrollo de Taylor de grado 3 de una función
$f$,  centrado en $a$ y en función de $x$ viene dado por:
\[
T_3 (f,a)(x) = f(a) + f'(a)(x - a) + \frac{{f''(a)}}{2}(x - a)^2  +
\frac{{f'''(a)}}{6}(x - a)^3
\]
y que el valor de la función en un cierto $x_0$, próximo a $a$,
puede calcularse de forma aproximada mediante:
\[
f(x_0 ) \approx T_3 (f,a)(x_0 )
\]
En nuestro caso, la función $f(x)= \tg x/2$ y debemos aproximar el
valor de $\tg 0.1$. Por lo tanto:
\[
\tg 0.1 = \tg \frac{{x_0 }}{2} \Leftrightarrow x_0  = 0.2
\]
Y como valor de $a$ (punto en el que centramos el polinomio de
Taylor) podemos tomar 0, ya que está lo suficientemente próximo a
$0.2$ como para que la aproximación sea buena, además de simplificar
notablemente los cálculos. Es decir, vamos a calcular el polinomio
de Maclaurin.

Entre las múltiples expresiones para la derivada de la tangente
(como cociente de senos y cosenos, con la secante y también con la
propia tangente), posiblemente la más cómoda para hacer derivadas de
orden superior es:
\[
f(x) = \tg (u(x)) \Rightarrow f'(x) = \left( {1 + \tg ^2 (u(x))}
\right)u'(x)
\]
Aplicado a nuestra función:
\[
f(x) = \tg \frac{x}{2}
\]
\[
f'(x) = \left( {1 + \tg ^2 \frac{x}{2}} \right)\frac{1}{2} =
\frac{1}{2} + \frac{1}{2}\tg ^2 \frac{x}{2}
\]

Procediendo de forma similar con las derivadas de segundo y tercer
orden, obtenemos:
\[
f''(x) = \frac{1}{2}\tg \frac{x}{2} + \frac{1}{2}\tg ^3
\frac{x}{2}
\]
\[
f'''(x) = \frac{1}{4} + \tg ^2 \frac{x}{2} + \frac{3}{4}\tg ^4
\frac{x}{2}
\]
Por lo tanto:
\[
f(0) = 0;\;f'(0) = 1/2;\;f''(0) = 0;\;f'''(0) = 1/4
\]
Con ello, el polinomio de Mac Laurin buscado es:
\[
T_3 (f,0)(x) = \frac{1}{2}x + \frac{1}{{24}}x^3
\]

Y la aproximación buscada vale:
\[
\tg 0.1 \approx T_3 (f,0)(0.2) = \frac{1}{2}\;0.2 +
\frac{1}{{24}}\;0.2^3  = 0.1003333
\]
Para comprobar que la aproximación obtenida es correcta, mediante
calculadora, utilizando como unidad angular el radián, obtenemos:
$\tg 0.1=0.1003346$


\item Sabemos que el error cometido con la anterior aproximación es
igual al valor absoluto del resto, y que este último viene dado por
la fórmula:
\[
R_n (f,a)(x) = \frac{{f^{(n + 1} (t)}}{{(n + 1)!}}(x - a)^{n + 1}
\]
donde $t$ pertenece al intervalo $(a,x)$ si $x>a$, o a $(x,a)$ si
$a>x$.
En nuestro caso:
\[
R_3 (f,0)(0.2) = \frac{{f^{(4} (t)}}{{4!}}(0.2)^{4}
\]
Si tenemos en cuenta que la derivada cuarta vale:
\[
f^{(4} (x) = \tg \frac{x}{2} + \frac{5}{2}\tg ^3 \frac{x}{2} +
\frac{3}{4}\tg ^5 \frac{x}{2} \Rightarrow f^{(4} (t) = \tg
\frac{t}{2} + \frac{5}{2}\tg ^3 \frac{t}{2} + \frac{3}{4}\tg ^5
\frac{t}{2}
\]
obtenemos:
\[
R_3 (f,0)(0.2) = \frac{{\tg \frac{t}{2} + \frac{5}{2}\tg ^3
\frac{t}{2} + \frac{3}{4}\tg ^5 \frac{t}{2}}}{{24}}\,(0.2)^4 ;\quad
t \in (0,\;0.2)
\]
Por lo tanto el error cometido vale:
\[
\left| {\frac{{\tg \frac{t}{2} + \frac{5}{2}\tg ^3 \frac{t}{2} +
\frac{3}{4}\tg ^5 \frac{t}{2}}}{{4!}}\,(0.2)^4 } \right|;\quad t
\in (0,\;0.2)
\]
Y nos piden que acotemos el error, es decir, que encontremos una
cierta cantidad tal que se demuestre que el error es menor o igual
que esa cantidad. Para ello, nos damos cuenta de que la función
tangente, e igualmente la tangente al cubo o a la quinta potencia,
son funciones crecientes en el intervalo $(0,\,0.2)$ y, por lo
tanto, el error alcanzará su máximo valor posible cuando $t$ sea un
valor muy próximo a $0.2$. No obstante obtenemos un error en cuya
expresión aparece de nuevo la tangente de $0.1$, y no tiene ningún
sentido utilizar la calculadora para calcular la tangente de $0.1$
presente en el resto, cuando precisamente es la tangente de $0.1$ lo
que pretendemos calcular mediante el polinomio de Taylor. No
obstante, podemos utilizar otras cotas menos precisas pero que
supongan cálculos fácilmente realizables sin necesidad de
calculadora, y evitando a la vez el que la tangente de $0.1$
aparezca en el error. Por ejemplo, la más sencilla se obtiene
considerando que en el intervalo $(0,\,0.2)$ $\tg(t/2)\leq 1$, e
igualmente la tangente cubo o a la quinta potencia.

Teniendo en cuenta lo anterior:
\[
{\rm Error} \le \frac{{(0.2)^4 }}{{4!}}\left| {1 + \frac{5}{2} +
\frac{3}{4}\,} \right| = 0.00033
\]
Una cota más precisa se obtiene tomando, por ejemplo, $t=\pi/3$, tal
que la tangente de $\pi/6$ sí que tiene un valor fácilmente
calculable (sin calculadora):
\[
\tg \frac{\pi }{6} = \frac{{1/2}}{{\sqrt 3 /2}} = \frac{{\sqrt 3
}}{3}
\]
Con ello, la cota para el error cometido vale:
\[
{\rm Error} \le \frac{{(0.2)^4 }}{{4!}}\left| {1\frac{{\sqrt 3 }}{3}
+ \frac{5}{2}\left( {\frac{{\sqrt 3 }}{3}} \right)^3  +
\frac{3}{4}\left( {\frac{{\sqrt 3 }}{3}} \right)^5 \,} \right| =
0.000077
\]
\end{enumerate}
}


\newproblem{tay-12}{gen}{*}
%ENUNCIADO
{Dada  la función $f(x) = 2\sqrt[3]{{1 + x}}$, se pide:
\begin{enumerate}
\item Hallar el polinomio de Maclaurin de tercer grado de la función.
\item Utilizar el polinomio obtenido en el apartado anterior para calcular un valor aproximado de $\sqrt[3]{9}$.
\item Dar una cota del error cometido.
\end{enumerate}
}
%SOLUCIÓN
{\begin{enumerate}
\item $P(3,f,0)(x)=2+\frac{2}{3}x-\frac{2}{9}x^2+\frac{10}{81}x^3$.
\item $P(3,f,0)(0.125)= 2.080102258$.
\item $|R_{3,f,0}(0.125)| \leq 2.00938\cdot 10^{-5}$.
\end{enumerate}
}
%RESOLUCIÓN
{\begin{enumerate}
\item   La fórmula del polinomio de Maclaurin de orden $3$ para la
    función $f$ es
\begin{equation}
    P_{3,f,0} (x)= f(0)+
    f'(0)x+\frac{f''(0)}{2!}x^2+\frac{f'''(0)}{3!}x^3,
    \label{Maclaurin}
\end{equation}
de modo que tenemos que calcular hasta la tercera derivada de $f$ en el 0.
\[ \renewcommand{\arraystretch}{2}
    \begin{array}{lll}
        f(x)=2\sqrt[3]{{1 + x}}=2(1+x)^{1/3}, & \quad \quad & f(0)=2(1+0)^{1/3}=2,  \\
        f'(x)=2\dfrac{1}{3}(1+x)^{-2/3}=\dfrac{2}{3}(1+x)^{-2/3}, &  & f'(0)=\dfrac{2}{3}(1+0)^{-2/3}=\dfrac{2}{3},\\
        f''(x)=\dfrac{2}{3}\dfrac{-2}{3}(1+x)^{-5/3}=\dfrac{-4}{9}(1+x)^{-5/3}, &  & f''(0)=\dfrac{-4}{9}(1+0)^{-5/3}=\dfrac{-4}{9},\\
        f'''(x)=\dfrac{-4}{9}\dfrac{-5}{3}(1+x)^{-8/3}=\dfrac{20}{27}(1+x)^{-8/3}, &  & f'''(0)=\dfrac{20}{27}(1+0)^{-8/3}=\dfrac{20}{27},
     \end{array}
\]
Sustituyendo estos valores en la ecuación \ref{Maclaurin},
obtenemos el polinomio que nos piden
\[
P(3,f,0)(x)= 2+\frac{2}{3}x+\frac{-4/9}{2}x^2+\frac{20/27}{6}x^3=
2+\frac{2}{3}x-\frac{2}{9}x^2+\frac{10}{81}x^3.
\]

\item Primero averiguamos en qué punto la función vale $\sqrt[3]{9}$.
\[
f(x)=2\sqrt[3]{1+x}=\sqrt[3]{9} \Leftrightarrow (2\sqrt[3]{1+x})^3=(\sqrt[3]{9})^3 \Leftrightarrow
2^3(1+x)=9 \Leftrightarrow 8+8x=9 \Leftrightarrow x=\frac{1}{8}=0.125.
\]
Calculando el polinomio anterior en este punto tenemos
\[
\sqrt[3]{9}\approx P(3,f,0)(0.125)= 2+\frac{2}{3}0.125-\frac{2}{9}0.125^2+\frac{10}{81}0.125^3=2.080102258.
\]

\item El error cometido en la aproximación anterior nos lo da el resto de Taylor, que en la forma
Lagrange es
\[
R_{3,f,0}(x)=\frac{f^{iv}(t)}{4!}x^4=\frac{\frac{160}{81} (1+t)^{-11/3}}{24}x^4 = \frac{20}{243}\frac{1}{(1+t)^{11/3}} x^4 \quad t\in(0,x),
\]
En el punto $x=0.125$ donde hemos calculado la aproximación, vale
\[
R_{3,f,0}(0.125)=\frac{20}{243}\frac{1}{(1+t)^{11/3}} 0.125^4= 2.00938\cdot 10^{-5}\frac{1}{(1+t)^{11/3}} \quad t\in(0\,,\,0.125).
\]
Para acotar el resto, basta con calcular el máximo de esta función en el
intervalo $(0\,,\,0.125)$. Puesto que la función $1/(1+t)^{11/3}$ es decreciente en dicho intervalo,
el máximo se alcanza en el extremo inferior del intervalo, es decir,
$t=0$. Así pues, tenemos la siguiente cota
\[
|R_{3,f,0}(0.125)|=|2.00938\cdot 10^{-5}\frac{1}{(1+t)^{11/3}}|\leq
|2.00938\cdot 10^{-5}\frac{1}{(1+0)^{11/3}}|=2.00938\cdot 10^{-5}.
\]
\end{enumerate}
}


\newproblem{tay-13}{gen}{*}
%ENUNCIADO
{Dada la función: $f(x) = \dfrac{2} {{\sqrt {3x+ 1} }}$
\begin{enumerate}
\item Obtener el polinomio de Mac Laurin de tercer grado.
\item Calcular el valor aproximado de $\dfrac{2}{1,3}$ empleando el polinomio anterior.
\end{enumerate}
}
%SOLUCIÓN
{\begin{enumerate}
\item $P(3,f,0)(x)= 2-3x+\frac{27}{4}x^2-\frac{135}{8}x^3$.
\item $\frac{2}{1.3}\approx P(3,f,0)(0.23)=1.461756910$.
\end{enumerate}
}
%RESOLUCIÓN
{\begin{enumerate}
\item   La fórmula del polinomio de Maclaurin de orden $3$ para la función $f$ es
\begin{equation}
P_{3,f,0} (x)= f(0)+ f'(0)x+\frac{f''(0)}{2!}x^2+\frac{f'''(0)}{3!}x^3,
\label{Maclaurin}
\end{equation}
de modo que tenemos que calcular hasta la tercera derivada de $f$ en el 0.
\[ \renewcommand{\arraystretch}{2}
    \begin{array}{ll}
        f(x)=\dfrac{2}{\sqrt{3x+1}}= 2(3x+1)^{-1/2}, & f(0)=2(3\cdot 0+1)^{-1/2}=2,  \\
        f'(x)=2\dfrac{-1}{2}(3x+1)^{-3/2}3=-3(3x+1)^{-3/2}, &  f'(0)=-3(3\cdot 0+1)^{-3/2}=-3,\\
        f''(x)=-3\dfrac{-3}{2}(3x+1)^{-5/2}3=\dfrac{27}{2}(3x+1)^{-5/2}, &  f''(0)=\dfrac{27}{2}(3\cdot 0+1)^{-5/2}=\dfrac{27}{2},\\
        f'''(x)=\dfrac{27}{2}\dfrac{-5}{2}(3x+1)^{-7/2}3=\dfrac{-405}{4}(3x+1)^{-7/2}, & f'''(0)=\dfrac{-405}{4}(3\cdot 0+1)^{-7/2}=\dfrac{-405}{4},
     \end{array}
\]
Sustituyendo estos valores en la ecuación \ref{Maclaurin}, obtenemos el polinomio que nos piden
\[
P(3,f,0)(x)= 2-3x+\frac{27/4}{2!}x^2-\frac{405/24}{3!}x^3=
2-3x+\frac{27}{4}x^2-\frac{135}{8}x^3.
\]

\item Primero averiguamos en qué punto la función vale $2/1.3$.
\[
f(x)=\frac{2}{\sqrt{3x+1}}=\frac{2}{1.3} \Leftrightarrow \sqrt{3x+1}=1.3 \Leftrightarrow
3x+1=1.3^2=1.69 \Leftrightarrow x=\frac{0.69}{3}=0.23.
\]
Calculando el polinomio anterior en este punto tenemos
\[
\frac{2}{1.3}\approx P(3,f,0)(0.23)= 2-3\cdot 0.23+\frac{27}{4}0.23^2-\frac{135}{8}0.23^3=1.461756910.
\]
\end{enumerate}
}


\newproblem*{tay-14}{gen}{*}
%ENUNCIADO
{Sea la función
\[
f(x) = 2\sqrt[4]{{1 + x}}
\]
\begin{enumerate}
\item Calcular su desarrollo de Maclaurin de orden 3.
\item Utilizar el desarrollo anterior para calcular de forma aproximada: $\sqrt[4]{{16,16}}$.
\item Acotar el error cometido con la aproximación anterior.
\end{enumerate}
}


\newproblem*{tay-15}{gen}{*}
%ENUNCIADO
{
Sea la función $f(x)=x^{x}.$
\begin{enumerate}
\item  Calcular su polinomio de Taylor de segundo orden, centrado en $x=1$.
\item  Aproximar con dicho polinomio el valor de: $1.1^{1.1}$.
\end{enumerate}
}


\newproblem*{tay-16}{gen}{*}
%ENUNCIADO
{Para la función:
\[
f(x)=\sqrt[3]{1+2x}
\]
Calcular:
\begin{enumerate}
\item  Su polinomio de Taylor de tercer orden centrado en 0.
\item  El valor aproximado de $\sqrt[3]{1.2}$ mediante el polinomio de Taylor calculado anteriormente.
\item  Acotar el error cometido mediante dicha aproximación.
\end{enumerate}
}


\newproblem*{tay-17}{gen}{*}
%ENUNCIADO
{Teniendo en cuenta que $\sen (2x)=2\sen x \cos x$ y los desarrollos de McLarin de las funciones $\sen x$ y $\cos x$, ¿cuál será el desarrollo de Maclaurin de orden 3 para la función $\sen (2x)$?

Calcular directamente el desarrollo de Maclaurin de orden 3 de la función $\sen (2x)$ para comprobar el resultado obtenido en el apartado anterior.
}


\newproblem*{tay-18}{gen}{*}
%ENUNCIADO
{En los libros de Cálculo se afirma que la función binómica, $(1+x)^p$, puede desarrollarse desarrollada mediante una serie de sumandos de la forma:
\[
\left( {1 + x} \right)^p  = 1 + px + \frac{{p(p - 1)}} {{2!}}x^2 + \frac{{p(p - 1)(p - 2)}} {{3!}}x^3  +  \cdots  + \frac{{p(p - 1) \cdots (p - k + 1)}} {{k!}}x^k
\]
para todo $x$ si $p$ es un entero no negativo.
\begin{enumerate}
\item Demostrar dicha fórmula hasta el término en $x^4$ mediante el desarrollo de Maclaurin de orden 4.
\item Utilizar el resultado anterior para calcular de forma aproximada el valor de: $0.9^6$.
\end{enumerate}
}


\newproblem{tay-19}{gen}{*}
%ENUNCIADO
{La concentración de un fármaco en sangre, en mg/dl, en función del tiempo, en horas, viene dada por la expresión:
\[
C(t) = \ln \left( {\frac{{t^2  + 2t + 1}}{{2t + 1}}} \right)
\]
Se pide:
\begin{enumerate}
\item Calcular el polinomio de Maclaurin de orden 3 de $C$.
\item Utilizar el polinomio anterior para dar el valor aproximado de la concentración al cabo de 15 minutos.
\item Calcular el polinomio de Taylor, centrado en 0, de orden 2 para la función:
\[
f(t) = 2\ln (t + 1) - \ln (2t + 1)
\]
\end{enumerate}
}
%SOLUCIÓN
{\begin{enumerate}
\item $P_{C,0}^3(t) = t^2-2t^3$.
\item $P_{C,0}^3(0.25) = 0.03125$.
\item $P_{f,0}^2(t) = t^2$.
\end{enumerate}}
%RESOLUCIÓN
{\begin{enumerate}
\item La ecuación del polinomio de Maclaurin de orden 3 de $C$ es
\begin{equation}
P_{C,0}^3(t) = C(0)+ C'(0)t + \frac{C''(0)}{2}t^2 + \frac{C'''(0)}{3!}t^3.
\end{equation}
Necesitamos calcular las 3 primeras derivadas, pero antes conviene simplificar la función
\begin{align*}
C(t)&= \ln\left(\frac{t^2+2t+1}{2t+2}\right)= \ln(t^2+2t+1)-\ln(2t+1) =\\
&= \ln((t+1)^2) -\ln(2t+1) = 2\ln(t+1)-\ln(2t+1),\\
C'(t)&= \frac{2}{t+1}-\frac{2}{2t+1},\\
C''(t)&= \frac{-2}{(t+1)^2}+\frac{4}{(2t+1)^2},\\
C'''(t)&= \frac{4}{(t+1)^3}-\frac{16}{(2t+1)^3}
\end{align*}
Las derivadas en 0 valen
\begin{align*}
C(0)&=  2\ln(1)-\ln(1) = 0,\\
C'(0)&= \frac{2}{1}-\frac{2}{1}= 0,\\
C''(0)&= \frac{-2}{1^2}+\frac{4}{1^2} = 2,\\
C'''(t)&= \frac{4}{1^3}-\frac{16}{1^3} = -12.
\end{align*}
Y sustituyendo en la ecuación anterior llegamos al polinomio
\[
P_{C,0}^3(t) = \frac{2}{2}t^2 - \frac{12}{3!}t^3 = t^2-2t^3.
\]

\item El valor de la función a los 15 minutos es $C(0.25)$ ya que las unidades del tiempo se consideran en horas. El valor aproximado de que da el polinomio en ese instante es
\[
P_{C,0}^3(0.25) = 0.25^2-2\cdot 0.25^3 = 0.03125.
\]

\item Según hemos podido comprobar al simplificar $C(t)$ resulta que $C(t)$ y $f(t)$ son la misma función, así que el polinomio de Maclaurin de orden 2 es el mismo que el calculado en el primer apartado pero considerando sólo hasta el término de grado 2, es decir,
\[
P_{f,0}^2(t) = t^2.
\]
\end{enumerate}
}


\newproblem{tay-20}{gen}{}
%ENUNCIADO
{Dada la función $\dfrac{\sen x+\cos x}{2}$:
\begin{enumerate}
\item  Utilizar el polinomio de Maclaurin de grado 3 para aproximar $\frac{\sen 1+\cos 1}{2}$.
\item  Utilizar el polinomio de Taylor de grado 2 en el punto $x_0 = \pi/2$ para aproximar $\frac{\sen 1+\cos 1}{2}$.
\item  Dar la cota de error cometida en ambas aproximaciones y decir cual es mejor.
\end{enumerate}
}
%SOLUCIÓN
{\begin{enumerate}
\item $P_0^3(x)=\frac{1}{2}+\frac{1}{2}x-\frac{1}{4}x^2-\frac{1}{12}x^3$ y $P_0^3(1)=\frac{2}{3}$.
\item $P_{\pi/2}^2(x)=\frac{1}{2}-\frac{1}{2}(x-\pi/2)-\frac{1}{4}(x-\pi/2)^2$ y  $P_{\pi/2}^2(1)= 0.70395$.
\item $|R_0^3(1)|\leq 0.04166$ y $|R_{\pi/2}^2(1)|\leq 0.03099$.
\end{enumerate}
}
%RESOLUCIÓN
{\begin{enumerate}
\item  El polinomio de Maclaurin de grado 3 para $f(x)$ viene dado por la fórmula siguiente:
\[
P_0^3(x)=f(0)+f^{\prime }(0)x+\frac{f^{\prime \prime }(0)}{2!}x^2+\frac{f^{\prime \prime \prime }(0)}{3!}x^3.
\]

Calculamos las tres primeras derivadas de $f(x)$:
\begin{align*}
f^{\prime }(x) &= \frac{\cos x-\sen x}{2}, \\
f^{\prime \prime }(x) &= \frac{-\ sen x-\cos x}{2}, \\
f^{\prime \prime \prime }(x) &= \frac{-\cos x+\sen x}{2}.
\end{align*}

Particularizando en $x=0$ tenemos:
\begin{align*}
f(0) &= \frac{\sen 0+\cos 0}2=\frac{1}{2}, \\
f^{\prime }(0) &= \frac{\cos 0-\sen 0}{2}=\frac{1}{2}, \\
f^{\prime \prime }(0) &= \frac{-\sen 0-\cos 0}{2}=-\frac{1}{2}, \\
f^{\prime \prime \prime }(0) &= \frac{-\cos 0+\sen 0}{2}=-\frac{1}{2}.
\end{align*}

Por tanto, el polinomio de Maclaurin que nos interesa es:
\[
P_0^3(x)=\frac{1}{2}+\frac{1}{2}x-\frac{1}{4}x^2-\frac{1}{12}x^3.
\]

Para aproximar $f(1)=\frac{\sen 1+\cos 1}{2}$, tenemos que tomar $x=1$, y en ese punto, la aproximación que da el polinomio es:
\[
P_0^3(1)=\frac{1}{2}+\frac{1}{2}-\frac{1}{4}-\frac{1}{12}=\frac{2}{3}.
\]

\item  El polinomio de Taylor de grado 2 en el punto $x_0=\pi/2$ para $f(x)$ viene dado por la fórmula siguiente:
\[
P_{\pi /2}^2(x)=f(\pi /2)+f^{\prime }(\pi /2)(x-\pi /2)+\frac{f^{\prime\prime }(\pi /2)}{2!}(x-\pi /2)^2.
\]

Particularizando en $x=\pi/2$ hasta la segunda derivada tenemos:
\begin{align*}
f(\pi /2) &= \frac{\sen\pi/2+\cos \pi /2}{2} = \frac{1}{2} \\
f^{\prime }(\pi/2) &= \frac{\cos \pi /2-\sen\pi/2}{2} = -\frac{1}{2}, \\
f^{\prime \prime}(\pi/2) &= \frac{-\sen\pi/2-\cos\pi/2}{2} = -\frac{1}{2}.
\end{align*}

Por tanto, el polinomio de Taylor que nos interesa es:
\[
P_{\pi/2}^2(x)=\frac{1}{2}-\frac{1}{2}(x-\pi/2)-\frac{1}{4}(x-\pi/2)^2,
\]

y, de nuevo, tomando $x=1$, la aproximación que da este polinomio para $f(1)=\frac{\sen 1+\cos 1}2$ es:
\[
P_{\pi/2}^2(1)=\frac 12-\frac{1}{2}(1-\pi/2)-\frac{1}{4}(1-\pi/2)^2 = 0.70395.
\]

\item  El error cometido en las aproximaciones se puede calcular mediante el resto de Lagrange. Para el polinomio de Maclaurin anterior, dicho resto se puede calcular con la fórmula siguiente:
\[
R_0^3(x) = \frac{f^{iv}(c)}{4!}x^4\qquad \mbox{con }c\in (0,x).
\]

Como la cuarta derivada de $f(x)$ coincide con $f(x)$, entonces particularizando el resto en $x=1$, tenemos:
\[
R_0^3(1)=\frac{\sen c+\cos c}{2\cdot 4!}1^4 = \frac{\sen c+\cos c}{48} \qquad \text{con }c\in (0,1),
\]
y puesto que tanto el seno como el coseno no pueden tomar valores mayores que 1, tenemos que $|\sen c+\cos c|\leq 2$, y obtenemos la siguiente cota de error para la primera aproximaci\'{o}n:
\[
|R_0^3(1)|\leq |\frac{2}{48}|=0.04166.
\]

Para el polinomio de Taylor, el resto de Lagrange tiene la forma siguiente:
\[
R_{\pi/2}^2(x)=\frac{f^{\prime \prime \prime }(c)}{3!}(x-\pi/2)^3\qquad \mbox{con }c\in (x,\pi /2).
\]

Como antes, particularizando en $x=1$ tenemos:
\[
R_{\pi/2}^2(1)=\frac{-\cos c+\sen c}{2\cdot 3!}(1-\pi/2)^3 \qquad \mbox{con }c\in (1,\pi /2),
\]
y como $|\sen c+\cos c|\leq 2,$ obtenemos la siguiente cota de error para la segunda aproximación:
\[
|R_{\pi/2}^2(1)|\leq |\frac{2}{2\cdot 3!}(1-\pi/2)^3|=0.03099.
\]

Podemos concluir que la segunda aproximación es mejor que la primera.
\end{enumerate}
}


\newproblem{tay-21}{gen}{*}
%ENUNCIADO
{Calcular el polinomio de Maclaurin de orden 4 de la función $\cos\frac{x}{3}$, y utilizarlo para aproximar $\cos\frac{\pi}{4}$, dando una
cota del error cometido.
}
%SOLUCIÓN
{$P_0^4(x)=1-\dfrac{1}{18}x^2+\dfrac{1}{1944}x^4$, $P_0^4\left(\dfrac{3\pi }{4}\right) = 0.7074292$ y la cota del error cometido es $\left|R_0^4\left(\dfrac{3\pi}{4}\right)\right| \leq 0.00249.$
}
%RESOLUCIÓN
{Llamando $f(x)=\cos\dfrac{x}{3}$, el polinomio que se nos pide viene dado por la fórmula siguiente:
\[
P_0^4(x)=f(0)+f^{\prime }(0)x+\frac{f^{\prime \prime }(0)}{2!}x^2+\frac{f^{\prime \prime \prime }(0)}{3!}x^3+\frac{f^{\text{iv}}(0)}{4!}x^4.
\]
Calculamos primero hasta la derivada cuarta de $f(x)$:
\[
\begin{array}{lll}
f(x)=\cos\dfrac{x}{3} & \qquad & f(0)=1 \\
f^{\prime }(x)=-\dfrac{1}{3}\sen\dfrac{x}{3} & &  f^{\prime }(0)=0 \\
f^{\prime \prime }(x)=-\dfrac{1}{9}\cos\dfrac{x}{3} &  & f^{\prime \prime }(0)=-\dfrac{1}{9}\\
f^{\prime \prime \prime}(x)=\dfrac{1}{27}\sen\dfrac{x}{3} & &  f^{\prime \prime \prime }(0)=0 \\
f^{iv}(x)=\dfrac{1}{81}\cos\dfrac{x}{3} & & f^{iv}(0)=\dfrac{1}{81}
\end{array}
\]
Sustituyendo estas derivadas en la fórmula anterior obtenemos el polinomio que buscamos:
\[
P_0^4(x)=1-\dfrac{1}{18}x^2+\frac{1}{1944}x^4.
\]
Para aproximar ahora $\cos\frac{\pi}{4}$ utilizando este polinomio, tenemos que calcular el valor del polinomio para un $x$ tal que $f(x)=\cos\frac{\pi}{4},$ es decir, un $x$ tal que $\cos\frac{x}{3}=\cos\frac{\pi}{4},$ de lo que se deduce $x=\frac{3\pi}{4}$. La aproximación que da el polinomio en este punto es
\[
P_0^4\left(\dfrac{3\pi }{4}\right)=1-\dfrac{1}{18}\left(\dfrac{3\pi}{4}\right)^2+\frac{1}{1944}\left(\dfrac{3\pi}{4}\right)^4=0.7074292
\]
Finalmente, el error cometido en esta aproximación lo da el resto de Langrange que se obtiene con la fórmula
\[
R_0^4\left(\dfrac{3\pi}{4}\right)=\frac{f^{v}(t)}{5!}\left(\dfrac{3\pi}{4}\right)^5\quad \text{con }t\in \left(0,\frac{3\pi}{4}\right).
\]
Calculando la quinta derivada de $f(t),$
\[
f^{v}(t)=-\dfrac{1}{243}\sen\dfrac{t}{3},
\]
y sustitutyendo en la fórmula anterior, obtenemos el error cometido:
\[
R_0^4\left(\dfrac{3\pi}{4}\right)=-\dfrac{\sen(t/3)}{29160}\left(\dfrac{3\pi}{4}\right)^5\quad \text{con }t\in \left(0,\frac{3\pi}{4}\right).
\]
Este error puede acotarse fácilemente aprovechando que $\left|\sen(t/3)\right| \leq 1,$ con lo que llegamos a la cota
\[
\left|R_0^4\left(\dfrac{3\pi}{4}\right)\right| \leq \left|-\dfrac{1}{29160}\left(\dfrac{3\pi}{4}\right)^5\right| = 0.00249.
\]
}


\newproblem{tay-22}{gen}{*}
%ENUNCIADO
{Calcular $0.98^{3/5}$ tomando hasta el término correspondiente a $n=3$ del desarrollo de Mac Laurin de la función $(1+x)^{3/5}$.
}
%SOLUCIÓN
{$P_{0}^{3}(x)=1+\dfrac{3}{5}x-\dfrac{3}{25}x^{2}+\dfrac{7}{125}x^{3}$ y $0.98^{3/5}\approx P_{0}^{3}(-0.02)= 0.9879515521$.
}
%RESOLUCIÓN
{Llamando $f(x)=(1+x)^{3/5}$, el polinomio que se nos pide viene dado por la fórmula siguiente:
\[
P_{0}^{3}(x)=f(0)+f^{\prime }(0)x+\frac{f^{\prime \prime }(0)}{2!}x^{2}+\frac{f^{\prime \prime \prime }(0)}{3!}x^{3}.
\]

Calculamos primero hasta la derivada tercera de $f(x)$ en el $0$:
\[
\renewcommand{\arraystretch}{2}
\begin{array}{lll}
f(x)=(1+x) ^{3/5} & \qquad & f(0)=1 \\
f^{\prime}(x)=\dfrac{3}{5}(1+x)^{-2/5} & & f^{\prime }(0)=\dfrac{3}{5}\\
f^{\prime \prime}(x)=-\dfrac{6}{25}(1+x)^{-7/5} & & f^{\prime \prime }(0)=-\dfrac{6}{25}\\
f^{\prime \prime \prime}(x)=\dfrac{42}{125}(1+x)^{-12/5} & & f^{\prime \prime \prime }(0)=\dfrac{42}{125}
\end{array}
\]

Sustituyendo estas derivadas en la fórmula anterior obtenemos el polinomio que buscamos:
\[
P_{0}^{3}(x)=1+\dfrac{3}{5}x-\dfrac{3}{25}x^{2}+\frac{7}{125}x^{3}.
\]

Para aproximar $0.98^{3/5}$ usando este polinomio, tenemos que calcular el valor del polinomio para un $x$ tal que $f(x)=0.98^{3/5}$, es decir, un $x$ tal que $0.98^{3/5}=(1+x)^{3/5}$, de lo que se deduce que $x=-0.02$. La aproximación que da el polinomio en este punto es
\[
P_{0}^{3}(-0.02)=1+\dfrac{3}{5}(-0.02)-\dfrac{3}{25}(-0.02)^{2}+\frac{7}{125}(-0.02)^{3}=0.9879515521.
\]
}


\newproblem{tay-23}{gen}{*}
%ENUNCIADO
{Obtener polinomio de Maclaurin de grado 3 de las funciones $\sen x$ y $\tg x$, y utilizar los polinomios anteriores para calcular
\[
\lim_{x\rightarrow 0}\frac{\tg x-x}{x-\sen x}
\]
}
%SOLUCIÓN
{$P_{0}^{3}(x)=0+1\cdot x+\frac{0}{2!}x^{2}+\frac{-1}{3!}x^{3}=x-\frac{x^{3}}{6}$,\\
$Q_{0}^{3}(x)=0+1\cdot x+\frac{0}{2!}x^{2}+\frac{2}{3!}x^{3}=x+\frac{x^{3}}{3}$ y \\
$\lim_{x\rightarrow 0}\frac{\tg x-x}{x-\sen x} = \lim_{x\rightarrow 0}\frac{x+\frac{x^{3}}{3}-x}{x-x+\frac{x^{3}}{6}} =2$.
}
%RESOLUCIÓN
{La formula general para calcular el polinomio de Maclaurin de grado 3 de una función $f(x)$ es:
\[
P_{0}^{3}(x)=f(0)+f^{\prime }(0)x+\frac{f^{\prime \prime }(0)}{2!}x^{2}+\frac{f^{\prime \prime \prime }(0)}{3!}x^{3}.
\]

Consideremos, en primer lugar, la función $\sen x$, y calculemos sus tres primeras derivadas en 0:
\[
\begin{array}{lll}
f(x)=\sen x &  & f(0)=\sen 0=0, \\
f^{\prime }(x)=\cos x &  & f^{\prime }(0)=\cos 0=1, \\
f^{\prime \prime }(x)=-\sen x &  & f^{\prime \prime }(0)=-\sen0=0, \\
f^{\prime \prime \prime }(x)=-\cos x &  & f^{\prime \prime \prime }(0)=-\cos 0=-1.
\end{array}
\]
Sustituyendo en la fórmula de arriba, llegamos al primer polinomio que buscamos:
\[
P_{0}^{3}(x)=0+1\cdot x+\frac{0}{2!}x^{2}+\frac{-1}{3!}x^{3}=x-\frac{x^{3}}{6}.
\]

Consideremos ahora la función $\tg x$ y calculemos sus tres primeras derivadas en 0:
\[
\begin{array}{lll}
g(x)=\tg x &  & g(0)=\tg 0=0 \\
g^{\prime }(x)=1+\tg^{2} x &  & g^{\prime }(0)=1+\tg^{2} 0=1 \\
g^{\prime \prime }(x)= 2\tg x+2\tg ^{3}x &  & g^{\prime \prime}(0)= 2\tg 0+2\tg ^{3}0=0 \\
g^{\prime \prime \prime }(x)=2+ 8\tg ^{2}x+6\tg ^{4}x &  & g^{\prime \prime \prime }(0)= 2+ 8\tg ^{2}0+6\tg ^{4}0=2
\end{array}
\]
Sustituyendo de nuevo en la fórmula de arriba, pero utilizando esta vez $g(x)$ en lugar de $f(x)$, llegamos al otro polinomio que buscamos:
\[
Q_{0}^{3}(x)=0+1\cdot x+\frac{0}{2!}x^{2}+\frac{2}{3!}x^{3}=x+\frac{x^{3}}{3}
\]

Finalmente, para calular ahora el límite que nos piden, podemos sustituir $\sen x$ por $P_{0}^{3}(x)$ y $\tg x$ por $Q_{0}^{3}(x)$, teniendo en cuenta dichos polinomios se comportan de igual forma que las correspondientes funciones en un entorno del 0. Así pues, tenemos:
\[
\lim_{x\rightarrow 0}\frac{\tg x-x}{x-\sen x} = \lim_{x\rightarrow 0}\frac{Q_{0}^{3}(x)-x}{x-P_{0}^{3}(x)} = \lim_{x\rightarrow 0}\frac{x+\frac{x^{3}}{3}-x}{x-x+\frac{x^{3}}{6}} = \lim_{x\rightarrow 0}\frac{\frac{x^{3}}{3}}{\frac{x^{3}}{6}} = \lim_{x\rightarrow 0}\frac{6}{3}=2.
\]
}
