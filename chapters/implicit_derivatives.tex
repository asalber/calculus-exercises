% Autor: Alfredo Sánchez Alberca (asalber@ceu.es)

\newproblem{derimp-1}{gen}{}
%STATEMENT
{In the following expressions, compute the derivative of $y$ as a function of $x$ using implicit differentiation techniques.
\begin{multicols}{2}
\begin{enumerate}
\item $x^3-3xy^2+y^3=1$.
\item $y=\dfrac{\sin(x+y)}{x^2+y^2}$.
\end{enumerate}
\end{multicols}
}
%SOLUTION
{\begin{enumerate}
\item $\frac{dy}{dx}=\frac{-x^2+y^2}{-2xy+y^2}.$
\item $\frac{dy}{dx}=\frac{-2xy+\cos(x+y)}{(x^2+y^2)+2y^2-\cos(x+y)}.$
\end{enumerate}
}
%RESOLUTION
{
}


\newproblem{derimp-2}{gen}{*}
%STATEMENT
{Compute the equations of the tangent and normal lines, at the point $x=0$, to the graph of the function $y$, given by $xy+e^x-\log y=0$.
}
%SOLUTION
{Tangent: $y=(e^2+e)x+e$. Normal: $y=\frac{-x}{e^2+e}+e$.
}
%RESOLUTION
{
}


\newproblem{derimp-3}{qui}{*}
%STATEMENT
{The temperature $T$ and the volume $V$ of a gas kept in a closed container of variable volume are related by the following formula,
where $T$ is given in Celsius degrees and $V$ in cubic meters:
\[
T^2(V^2-\pi^2)-V\cos(TV)=0
\]
\begin{enumerate}
\item Compute the derivative of the volume with respect to the temperature when the volume is equal to $\pi$ m$^3$ and the temperature
is equal to half Celsius degree.
\item Compute an equation of the line tangent to the graph of the volume, as a function of the temperature, in the point of part (a).
\item If we assume that temperature and volume are both functions of the pressure, find the equation that would relate the derivative of the
temperature (with respect to pressure) with the derivative of the volume (with respect to pressure).
\end{enumerate}
}
%SOLUTION
{\begin{enumerate}
\item $\frac{dV}{dT} = \frac{-2T(V^2-\pi^2)-V^2\sen(TV)}{2T^2V-\cos(TV)+TV\sen(TV)}$ and $\frac{dV}{dT}(V=\pi,T=0.5)= -\pi$ m$^3$/ºC.
\item Tangent: $V=\pi(-T+1.5)$.
\item $2T\frac{dT}{dP}(V^2-\pi^2)+T^2(2V\frac{dV}{dT})-\frac{dV}{dT}\cos(VT)-V(-\sen(TV)(\frac{dT}{dP}V+T\frac{dV}{dP})) =0$.
\end{enumerate}
}
%RESOLUTION
{
}


\newproblem*{derimp-4}{gen}{*}
%STATEMENT
{Dada la función:
\[
\frac{x}{{y^2 }} + e^{x^2 y}  - \log \sqrt {x - y}  = 0
\]
Calcular las ecuaciones de las rectas tangente y normal a su gráfica en el punto de abcisa $x=0$.
}


\newproblem{derimp-5}{gen}{*}
%STATEMENT
{Un cuerpo se mueve en el plano a través de los puntos de coordenadas $(x,y)$ relacionadas mediante la siguiente expresión:
\[
2e^{xy} \sen x + y\cos x = 2
\]
Se pide:
\begin{enumerate}
\item Calcular su posición cuando $x=\pi/2$
\item Calcular la ecuación de la recta tangente a la gráfica de la función cuando $x=0$.
\end{enumerate}
}
%SOLUTION
{
\begin{enumerate}
\item $x_0=\pi/2$, $y_0 = 0$.
\item $ y=-2x+2$.
\end{enumerate}
}
%RESOLUTION
{\begin{enumerate}
\item El punto $(x_0,y_0)$ en el que se encontrará el cuerpo cuando $x_0=\pi/2$
cumple la ecuación del enunciado. Por lo tanto:
\[
2e^{x_0y_0} \sen x_0 + y_0\cos x_0 = 2
\]
y sustituyendo $x_0=\pi/2$, obtenemos:
\[
2e^{\frac{\pi }{2}y_0 } \sen\frac{\pi }{2} + y_0 \cos \frac{\pi }{2}
= 2 \Rightarrow 2e^{\frac{\pi }{2}y_0 }  = 2
\]
con lo cual:
\[
e^{\frac{\pi }{2}y_0 }  = 1 \Leftrightarrow y_0  = 0
\]

\item Sabemos que la ecuación de la recta tangente a la gráfica de
la función en el punto de coordenadas $(x_0,y_0)$ es:
\[
y-y_0=f'(x_0)(x-x_0)
\]
En nuestro caso, no tenemos la expresión explícita de la función
$f$, pero sí que tenemos la expresión de partida que define a $y$
como función de $x$ de forma implícita. Suponiendo que $y$ es
función de $x$ y derivando con respecto a $x$, obtenemos:
\[
2 e^{xy}(y + xy')\sen x + 2e^{xy}\cos x + y'\cos x - y\sen x = 0
\]
Y sacando como factor común $y'$ y despejando, nos queda:
\[
y' = \frac{{ - 2e^{xy} y \sen x - 2e^{xy} \cos x + y \sen
x}}{{2xe^{xy} \sen x + \cos x}}
\]
Teniendo en cuenta que $x_0=0$, y sustituyendo en la expresión
inicial, obtenemos el valor de $y_0$:
\[
2e^0 \sen 0 + y_0 \cos 0 = 2 \Rightarrow y_0  = 2
\]
Y teniendo en cuenta el valor de la derivada implícita obtenido
anteriormente:
\[
f'(0) = y'(0,2) = \frac{{ - 2e^0 \cos 0}}{{\cos 0}} =  - 2
\]
Por lo tanto, la ecuación de la recta tangente es:
\[
y-2=-2(x-0)\Leftrightarrow y=-2x+2
\]
\end{enumerate}
}


\newproblem{derimp-6}{gen}{*}
%STATEMENT
{Hallar la ecuación de la recta tangente y normal a la curva $x^2+y^2=3xy-1$ en los puntos en que $x=1$. Calcular también los extremos relativos y decir si son máximos o mínimos.
}
%SOLUTION
{Recta tangente en el punto $(1,1)$: $y= 2-x$. Recta normal en el punto $(1,1)$: $y=x$.

Recta tangente en el punto $(1,2)$: $y= 4x-2$. Recta normal en el punto $(1,2)$: $y=\frac{9-x}{4}$.

Hay un mínimo relativo en el punto $(3/\sqrt{5},2/\sqrt{5})$ y un máximo relativo en $(-3/\sqrt{5},-2/\sqrt{5})$.
}
%RESOLUTION
{Consideremos $y$ como función de $x$. Veamos primero los puntos de la curva en los que $x=1$:
\[
1^{2}+y^{2} = 3\cdot 1\cdot y-1 \Leftrightarrow y^{2} -3y+2 = 0.
\]
Resolviendo la ecuación obtenemos dos soluciones $y=1$ e $y=2$, de modo que existen dos puntos para los que $x=1$, que son el $(1,1)$ y el $(1,2)$.

Para calcular las ecuaciones de las rectas tangente y normal en estos puntos, necesitamos calcular la derivada $dy/dx$ en dichos puntos. Derivamos implicitamente:
\[
d(x^{2}+y^{2} = d(3xy-1) \Leftrightarrow 2xdx+2ydy = 3(dxy+xdy) \Leftrightarrow (2x-3y)dx+(2y-3x)dy = 0,
\]
de donde se deduce
\[
\frac{dy}{dx} = \frac{3y-2x}{2y-3x},
\]
que en el punto $(1,1)$ vale
\[
\frac{dy}{dx} = \frac{3\cdot 1 - 2\cdot 1}{2\cdot 1-3\cdot 1} = -1,
\]
y en el punto $(1,2)$ vale
\[
\frac{dy}{dx} = \frac{3\cdot 2 - 2\cdot 1}{2\cdot 2-3\cdot 1} = 4.
\]
Así pues, la ecuación de la recta tangente en el punto $(1,1)$ es
\[
y-1 = \frac{dy}{dx}(1,1)(x-1) \Leftrightarrow y= 2-x,
\]
y la ecuación de la recta normal es
\[
y-1 = -\frac{1}{dy/dx}(1,1)(x-1) \Leftrightarrow y= x,
\]
mientras que la ecuación de la recta tangente en el punto $(1,2)$ es \[
y-2 = \frac{dy}{dx}(1,2)(x-1) \Leftrightarrow y= 4x-2,
\]
y la ecuación de la recta normal es
\[
y-2 = -\frac{1}{dy/dx}(1,2)(x-1) \Leftrightarrow y=\frac{9-x}{4}.
\]

Por otro lado, para calcular los extremos relativos, primero calculamos los puntos críticos, que son los que anulan la primera derivada:
\[
\frac{dy}{dx}=\frac{3y-2x}{2y-3x}=0 \Leftrightarrow 3y-2x=0 \Leftrightarrow y=2x/3.
\]
Pero además deben pertenecer a la curva de la función, y por tanto deben satisfacer la ecuación de la función:
\[
x^{2}+(2x/3)^{2}=3x(2x/3)-1 \Leftrightarrow x^{2}+4x^{2}/9 = 2x^{2}-1 \Leftrightarrow x^{2}=9/5 \Leftrightarrow x=\pm 3/\sqrt{5}.
\]
Así pues, existen dos puntos críticos que son el $(3/\sqrt{5},2/\sqrt{5})$ y $(-3/\sqrt{5},-2/\sqrt{5})$. Para ver si son puntos de máximo o mínimo relativos, necesitamos calcular la segunda derivada en dichos puntos:
\[
\frac{d^{2}y}{dx^{2}}= \frac{d}{dx}\left(\frac{3y-2x}{2y-3x}\right) = \frac{(3\frac{dy}{dx}-2)(2y-3x)-(2\frac{dy}{dx}-3)(3y-2x)}{(2y-3x)^{2}}.
\]
En el punto, $(3/\sqrt{5},2/\sqrt{5})$ la segunda derivada vale
\[
\frac{d^{2}y}{dx^{2}}(3/\sqrt{5},2/\sqrt{5})=
\frac{(3\cdot 0-2)(2\frac{2}{\sqrt{5}}-3\frac{3}{\sqrt{5}})-(2\cdot 0-3)(3\frac{2}{\sqrt{5}}-2\frac{3}{\sqrt{5}})}{(2\frac{2}{\sqrt{5}}-3\frac{3}{\sqrt{5}})^{2}} = 2\sqrt{5},
\]
que al ser positiva, indica que el punto $(3/\sqrt{5},2/\sqrt{5})$ es un punto de mínimo relativo.

En el punto, $(-3/\sqrt{5},-2/\sqrt{5})$ la segunda derivada vale
\[
\frac{d^{2}y}{dx^{2}}(-3/\sqrt{5},-2/\sqrt{5})=
\frac{(3\cdot 0-2)(2\frac{-2}{\sqrt{5}}-3\frac{-3}{\sqrt{5}})-(2\cdot 0-3)(3\frac{-2}{\sqrt{5}}-2\frac{-3}{\sqrt{5}})}{(2\frac{-2}{\sqrt{5}}-3\frac{--3}{\sqrt{5}})^{2}} = -2\sqrt{5},
\]
que al ser negativa, indica que el punto $(-3/\sqrt{5},-2/\sqrt{5})$ es un punto de máximo relativo.
}


\newproblem{derimp-7}{gen}{*}
%STATEMENT
{Dada  la curva $x^2-xy+y^2=3$
\begin{enumerate}
\item Calcular los posibles extremos relativos de $y$, considerando $y$ como función implícita de $x$. ¿En qué puntos se alcanzan dichos valores?
\item Analizar si lo puntos anteriores son máximos o mínimos haciendo uso de la derivada segunda.
\end{enumerate}
}
%SOLUTION
{\begin{enumerate}
\item Los puntos donde se anula la derivada son $(1,2)$ y $(-1,-2)$.
\item En $(1,2)$ hay un máximo y en $(-1,-2)$ un mínimo.
\end{enumerate}
}
%RESOLUTION
{\begin{enumerate}
\item Derivamos implícitamente la ecuación
\[
\frac{d}{dx}(x^2-xy+y^2)=\frac{d}{dx}3=0
\]
Derivando el lado izquierdo tenemos
\begin{align*}
\frac{d}{dx}(x^2-xy+y^2)&=
\frac{d}{dx}(x^2)-\frac{d}{dx}(xy)+\frac{d}{dx}(y^2)=
2x-(\frac{dx}{dx}y+x\frac{dy}{dx})+2y\frac{dy}{dx}=\\
&=2x-y-x\frac{dy}{dx}+2y\frac{dy}{dx}
=2x-y+(2y-x)\frac{dy}{dx}=0
\end{align*}
Los posibles extremos serán los puntos donde se anule la derivada, es decir, $\dfrac{dy}{dx}=0$. Sustituyendo en la ecuación anterior tenemos
\[
2x-y+(2y-x)\cdot 0= 2x-y=0 \Leftrightarrow y=2x.
\]
Y sustituyendo ahora en la ecuación de la función tenemos
\[
x^2-x\cdot 2x+(2x)^2=3 \Leftrightarrow x^2-2x^2+4x^2=3 \Leftrightarrow 3x^2=3 \Leftrightarrow x=\pm1.
\]
Por tanto, los posibles puntos de extremo serán (1,2) y (-1,-2).

\item Para ver si los puntos anteriores son efectivamente extremos, calculamos la derivada segunda en dichos puntos.
\begin{align*}
\frac{d^2}{dx^2}(x^2-xy+y^2)&=
\frac{d}{dx}\left(\frac{d}{dx}(x^2-xy+y^2)\right)=\\
&=\frac{d}{dx}\left(2x-y+(2y-x)\frac{dy}{dx}\right)=\\
&=\frac{d}{dx}(2x)-\frac{d}{dx}y+\left(\frac{d}{dx}(2y-x)\frac{dy}{dx}+(2y-x)\frac{d}{dx}\frac{dy}{dx}\right)=\\
&=2-\frac{dy}{dx}+\left(2\frac{dy}{dx}-\frac{dx}{dx}\right)\frac{dy}{dx}+(2y-x)\frac{d^2y}{dx^2}=\\
&= 2-\frac{dy}{dx}+2\frac{d^2y}{dx^2}-\frac{dy}{dx}+(2y-x)\frac{d^2y}{dx^2}=
2-2\frac{dy}{dx}+(2y-x+2)\frac{d^2y}{dx^2}=0
\end{align*}
Para el primer punto tenemos que sustituir $x=1$, $y=2$ y $\dfrac{dy}{dx}=0$, y queda
\[
2-2\cdot 0+(2\cdot 2-1+2)\frac{d^2y}{dx^2}=0 \Leftrightarrow 2+5\frac{d^2y}{dx^2}=0 \Leftrightarrow \frac{d^2y}{dx^2}=\frac{-2}{5},
\]
que al ser negativo indica que el en el punto $(1,2)$ hay un máximo.

Para el segundo punto tenemos que sustituir $x=-1$, $y=-2$ y $\dfrac{dy}{dx}=0$, y queda
\[
2-2\cdot 0+(2\cdot (-2)-(-1)1+2)\frac{d^2y}{dx^2}=0 \Leftrightarrow 2-\frac{d^2y}{dx^2}=0 \Leftrightarrow \frac{d^2y}{dx^2}=2,
\]
que al ser positivo indica que el en el punto $(-1,-2)$ hay un mínimo.
\end{enumerate}
}


\newproblem{derimp-8}{gen}{}
%STATEMENT
{Caldular $dy/dx$ y $dx/dy$ para las siguientes funciones implícitas:
\begin{enumerate}
\item $2x^2+3y^3-x^y= 2$
\item $3x^2y^2 = x^2 + 3y^3$
\item $\sen (xy^2) = \cos(x^2y)$
\item $x \ln (x^2+3y)  = y^3e^{2xy^2}$
\end{enumerate}
}
%SOLUTION
{\begin{enumerate}
\item $\frac{dy}{dx} = \frac{4x-yx^{y-1}}{-9y^2+x^y\log_x e}$ y $\frac{dx}{dy} = \frac{-9y^2+x^y\log_x e}{4x-yx^{y-1}}$.
\item $\frac{dy}{dx} = \frac{6xy^2-2x}{9y^2-6x^2y}$ y $\frac{dx}{dy} = \frac{9y^2-6x^2y}{6xy^2-2x}$.
\item $\frac{dy}{dx} = \frac{y^2\cos(xy^2)+2xy\sen(x^2y)}{-2xy\cos(xy^2)-x^2\sen(x^2y)}$ y $\frac{dx}{dy} = \frac{-2xy\cos(xy^2)-x^2\sen(x^2y)}{y^2\cos(xy^2)+2xy\sen(x^2y)}$.
\item $\frac{dy}{dx} = \frac{\ln(x^2+3y)+\frac{2x^2}{x^2+3y}-2y^5e^{2xy^2}}{-\frac{3x}{x^2+3y}3y^2e^{2xy^2}+4xy^4e^{2xy^2}}$ y $\frac{dx}{dy} = \frac{-\frac{3x}{x^2+3y}3y^2e^{2xy^2}+4xy^4e^{2xy^2}}{\ln(x^2+3y)+\frac{2x^2}{x^2+3y}-2y^5e^{2xy^2}}$.
\end{enumerate}
}
%RESOLUTION
{
}


\newproblem*{derimp-9}{gen}{}
%STATEMENT
{Caldular $dy/dx$ para las siguientes funciones definidas implícitamente:
\begin{enumerate}
\item  $xy^{3}-3x^{2}=xy+5.$
\item  $e^{xy}+y\log x=\cos 2x.$
\end{enumerate}
}


\newproblem*{derimp-10}{gen}{*}
%STATEMENT
{La expresión
\[
e^{xy}\log \left( \dfrac{1}{x}\right) +a\dfrac{1}{y}=2,
\]
donde $a$ es una constante, define a $y$ como función implícita de $x $ en el punto $\left( x_{0},y_{0}\right) =\left( 1,1\right) .$
Calcular la derivada de $y$ con respecto a $x$ en dicho punto.
}

\newproblem{derimp-11}{gen}{*}
%STATEMENT
{La concentración de un fármaco en sangre, $C$ en mg/dl, y el tiempo, $t$ en s,
están relacionados mediante la expresión:
\[
e^{tC}-t^2C^3-\ln{C}=0
\]
Suponiendo que la ecuación anterior define a $C$ como función implícita de $t$, y que, por lo tanto, también puede
definir a $t$ como función implícita de $C$, calcular:
\begin{enumerate}
\item La derivada de $C$ con respecto a $t$.
\item La ecuación de la recta tangente a la gráfica de $C$ en función de $t$ cuando $t=0$.
\item La ecuación de la recta normal a la gráfica de $t$ en función de $C$ cuando $C=e$.
\end{enumerate}
}
%SOLUTION
{
\begin{enumerate}
\item $\displaystyle
\dfrac{dC}{dt}=\dfrac{2tC^4-e^{tC}C^2}{e^{tC}tC-3t^2C^3-1}$.
\item $C=e+e^2 t$.
\item $t=-e^2(C-e)$.
\end{enumerate}
}
%RESOLUTION
{\begin{enumerate}
\item Derivamos implícitamente:
\[
\renewcommand{\arraystretch}{2}
\begin{array}{c}
d(e^{tC}-t^2C^3-\ln{C})=d0 \Leftrightarrow\\
\Leftrightarrow d(e^{tC})-d(t^2C^3)-d\ln{C}= 0 \Leftrightarrow\\
\Leftrightarrow e^{tC}d(tC)-(d(t^2)C^3+t^2d(C^3))-\dfrac{1}{C}dC= 0 \Leftrightarrow\\
\Leftrightarrow e^{tC}(dtC+tdC)-2tdtC^3-t^23C^2dC-\dfrac{1}{C}dC= 0 \Leftrightarrow\\
\Leftrightarrow e^{tC}Cdt+e^{tC}tdC-2tC^3dt-3t^2C^2dC-\dfrac{1}{C}dC= 0 \Leftrightarrow\\
\Leftrightarrow e^{tC}C^2dt+e^{tC}tCdC-2tC^4dt-3t^2C^3dC-dC= 0 \Leftrightarrow\\
\Leftrightarrow (e^{tC}C^2-2tC^4)dt+(e^{tC}tC-3t^2C^3-1)dC= 0 \Leftrightarrow\\
\Leftrightarrow (e^{tC}tC-3t^2C^3-1)dC= (2tC^4-e^{tC}C^2)dt \Leftrightarrow\\
\Leftrightarrow \dfrac{dC}{dt}=\dfrac{2tC^4-e^{tC}C^2}{e^{tC}tC-3t^2C^3-1}.
\end{array}
\]

\item En primer lugar veamos qué valores de $C$ corresponden a $t=0$. Sustituyendo en la ecuación que define la función tenemos:
\[
e^{0\cdot C}-0^2C^3-\ln C = 0 \Leftrightarrow 1-\ln C = 0 \Leftrightarrow \ln C = 1 \Leftrightarrow C = e.
\]
Así pues, se trata de calcular la ecuación de la recta tangente en el punto $(t=0, C=e)$. La ecuación de la recta tangente es
\[
C = e+\dfrac{dC}{dt}(t=0,C=e)(t-0)
\]
Utilizando la derivada calculada en el apartado anterior tenemos
\[
\dfrac{dC}{dt}(t=0,C=e)=\dfrac{2\cdot 0\cdot e^4-e^{0\cdot e}e^2}{e^{0\cdot e}\cdot 0\cdot e-3\cdot 0^2\cdot e^3-1} = \dfrac{-e^2}{-1} = e^2,
\]
y sustituyendo en la ecuación de la tangente llegamos a
\[
C= e+e^2 t.
\]

\item Del apartado anterior se deduce que el valor de $t$ que le corresponde a $C=e$ según la función es $t=0$, es decir, se trata de calcular la recta normal en el mismo punto del apartado anterior pero considerando a $t$ como función de $C$. La ecuación de la recta normal es
\[
t = 0-\dfrac{1}{dt/dC}(t=0,C=e)(C-e) = -\dfrac{dC}{dt}(t=0,C=e)(C-e),
\]
y como la derivada ya la tenemos del apartado anterior, simplemente sustituimos y tenemos
\[
t= -e^2(C-e).
\]
\end{enumerate}
}


\newproblem{derimp-12}{gen}{*}
{Calcular las ecuaciones de las tangentes a la curva definida por
\[
\tg(xy) -\cos\left(\frac{x}{y}\right) +\frac{x}{y^2} = \log (x^2+y^2)
\]
en los puntos de abscisa $x=0$.
}
%SOLUTION
{Ecuación de la recta tangente en $(0,-\sqrt{e^{-1}})$: $y = -\sqrt{e^{-1}} + \frac{1-e\sqrt{e}}{2e} x$.\\
Ecuación de la recta tangente en el punto  $(0,\sqrt{e^{-1}})$: $y = \sqrt{e^{-1}} + \frac{1+e\sqrt{e}}{2e} x$.
}
%RESOLUTION
{En primer lugar hay que calcular los puntos de la función en los que $x=0$, es decir los puntos donde la función corta el eje de ordenadas. Sustituyendo $x$ por 0 en la ecuación que define la función tenemos
\[
\tg(0) -\cos 0 + 0 = \log (y^2) \Leftrightarrow \log(y^2) = -1
\Leftrightarrow y^2 = e^{-1} \Leftrightarrow y=\pm \sqrt{e^{-1}}
\]
Así pues, hay dos puntos en los que hay que calcular las tangentes que son
$(0,-\sqrt{e^{-1}})$ y $(0,\sqrt{e^{-1}})$.

La ecuación de la recta tangente en $(0,-\sqrt{e^{-1}})$ es
\begin{equation}
y = -\sqrt{e^{-1}} + \frac{dy}{dx}(0,-\sqrt{e^{-1}})(x-0),
\end{equation}
y por tanto necesitamos calcular la derivada $\dfrac{dy}{dx}$. Para ello derivamos implícitamente:
\[
\renewcommand{\arraystretch}{2.5}
\begin{array}{c}
\displaystyle
d\left(\tg(xy)-\cos\left( \frac{x}{y}\right)+\frac{x}{y^2}\right) = d\left( \log (x^2+y^2)\right) \Leftrightarrow \\ \displaystyle
\Leftrightarrow
d\tg(xy)-d\cos\left(\frac{x}{y}\right)+d\left( \frac{x}{y^2}\right)=
\frac{1}{x^2+y^2}d(x^2+y^2) \Leftrightarrow \\ \displaystyle
\Leftrightarrow
(1+\tg^2(xy))d(xy)+\sen\left( \frac{x}{y}\right)d\left( \frac{x}{y}\right) +\frac{dxy^2-xdy^2}{y^4} =
\frac{1}{x^2+y^2}(dx^2+dy^2) \Leftrightarrow \\ \displaystyle
\Leftrightarrow
(1+\tg^2(xy))(dxy+xdy)+\sen\left( \frac{x}{y}\right) \frac{dxy-xdy}{y^2}+\frac{dxy^2-x2ydy}{y^4} =
\frac{1}{x^2+y^2}(2xdx+2ydy) \Leftrightarrow \\ \displaystyle
\Leftrightarrow
(1+\tg^2(xy))(dxy+xdy)+\sen\left(\frac{x}{y}\right)\frac{dxy-xdy}{y^2}+\frac{dxy-2xdy}{y^3} =
\frac{2xdx+2ydy}{x^2+y^2}
\end{array}
\]
La derivada en el punto $(0,-\sqrt{e^{-1}})$ es
\[
\renewcommand{\arraystretch}{2.5}
\begin{array}{c}\displaystyle
(1+\tg^2(0\cdot
(-\sqrt{e^{-1}})))(dx(-\sqrt{e^{-1}})+0dy)+\sen\left(\frac{0}{-\sqrt{e^{-1}}}\right)\frac{dx(-\sqrt{e^{-1}})-0dy}{e^{-1}}+\frac{dx(-\sqrt{e^{-1}})-2\cdot 0dy}{(-\sqrt{e^{-1}})^3} =\\
\displaystyle
= \frac{2\cdot 0dx+2\cdot (-\sqrt{e^{-1}})dy}{0^2+(-\sqrt{e^{-1}})^2}
\Leftrightarrow\\ \displaystyle
\Leftrightarrow -\sqrt{e^{-1}}dx+\frac{-\sqrt{e^{-1}}dx}{(-\sqrt{e^{-1}})^3} = \frac{-2\sqrt{e^{-1}}dy}{e^{-1}} \Leftrightarrow\\ \displaystyle
\Leftrightarrow -\sqrt{e^{-1}}dx+\frac{dx}{e^{-1}} =
-2\sqrt{e}dy \Leftrightarrow\\ \displaystyle
\Leftrightarrow -\sqrt{e^{-1}}dx+edx = -2\sqrt{e}dy \Leftrightarrow\\
\displaystyle \Leftrightarrow (e-\sqrt{e^{-1}})dx = -2\sqrt{e}dy
\Leftrightarrow\\
\displaystyle \Leftrightarrow \frac{dy}{dx} =
\frac{e-\sqrt{e^{-1}}}{-2\sqrt{e}} = \frac{1-e\sqrt{e}}{2e}.
\end{array}
\]
y sustituyendo en la ecuación anterior tenemos que la tangente en el punto  $(0,-\sqrt{e^{-1}})$ es
\[
y = -\sqrt{e^{-1}} + \frac{1-e\sqrt{e}}{2e} x,
\]

Del mismo modo, la derivada en el punto $(0,-\sqrt{e^{-1}})$ es
\[
\renewcommand{\arraystretch}{2.5}
\begin{array}{c} \displaystyle
(1+\tg^2(0\cdot \sqrt{e^{-1}}))(dx\sqrt{e^{-1}}+0dy)+\sen\left(\frac{0}{\sqrt{e^{-1}}}\right)\frac{dx\sqrt{e^{-1}}-0dy}{e^{-1}}+\frac{dx\sqrt{e^{-1}}-2\cdot 0dy}{\sqrt{e^{-1}}^3}= \\ \displaystyle
= \frac{2\cdot 0dx+2\cdot\sqrt{e^{-1}}dy}{0^2+\sqrt{e^{-1}}^2} \Leftrightarrow\\ \displaystyle
\Leftrightarrow \sqrt{e^{-1}}dx+\frac{\sqrt{e^{-1}}dx}{\sqrt{e^{-1}}^3} =
\frac{2\sqrt{e^{-1}}dy}{e^{-1}} \Leftrightarrow\\ \displaystyle
\Leftrightarrow \sqrt{e^{-1}}dx+\frac{dx}{e^{-1}} =
2\sqrt{e}dy \Leftrightarrow\\ \displaystyle
\Leftrightarrow \sqrt{e^{-1}}dx+edx = 2\sqrt{e}dy \Leftrightarrow\\ \displaystyle
\Leftrightarrow (e+\sqrt{e^{-1}})dx = 2\sqrt{e}dy \Leftrightarrow\\ \displaystyle
\Leftrightarrow \frac{dy}{dx} = \frac{e+\sqrt{e^{-1}}}{2\sqrt{e}} = \frac{1+e\sqrt{e}}{2e}.
\end{array}
\]
y la ecuación de la tangente en el punto  $(0,\sqrt{e^{-1}})$ es
\[
y = \sqrt{e^{-1}} + \frac{1+e\sqrt{e}}{2e} x,
\]
}


\newproblem{derimp-13}{gen}{}
%STATEMENT
{Compute the equations of the tangent and normal lines to the curve $C$ at the point $P$ given below:
\begin{enumerate}
\item $\displaystyle C:\frac{x^2}{9}-\frac{y^2}{4}=1$, $P=(-3,0)$.
\item $C:x^3-y^5+xy^2 = 8$, $P=(2,0)$.
\item $C:x=y^2$, $P=(0,0)$.
\item $C:x^{2/3}+y^{2/3}=1$, $P=(\sqrt2/4,\sqrt2/4)$.
\end{enumerate}
}
%SOLUTION
{\begin{enumerate}
\item Tangent $x=-3$ and normal $y=0$.
\item Tangent $x=2$ and normal $y=0$.
\item Tangent $x=0$ and normal $y=0$.
\item Tangent $y=-x+\sqrt{2}/2$ and normal $y=x$.
\end{enumerate}
}
%RESOLUTION
{
}


\newproblem{derimp-14}{gen}{}
%STATEMENT
{On each of the following cases below, find the equation of the tangent plane and the normal line to the surface $S$ at the given point
$P$:
\begin{enumerate}
\item $S:x-y+z=1$, $P=(0,0,1)$.
\item $S:x^2+y^2+z^2=1$, $P=(0,1,0)$.
\item $S:z=\log(x^2+y^2)$, $P=(1,0,0)$.
\item $S:z=e^{-(x^2+y^2)}$, $P=(0,0,1)$.
\item $S:z=e^{x+y}\sin x$, $P=(\pi,0,0)$.
\end{enumerate}
}
%SOLUTION
{\begin{enumerate}
\item Tangent plane $x-y+z-1=0$ and normal line $\frac{x}{-1}=\frac{y}{1}=\frac{z-1}{-1}$.
\item Tangent plane $y=1$ and normal line $(x=0, y=1+2t, z=0)$.
\item Tangent plane $2x-z-2=0$ and normal line $(x=1+2t, y=0, z=-t)$.
\item Tangent plane $z=1$ and normal line $(x=0, y=0, z=1+t)$.
\item Tangent plane $z=-e^\pi(x-\pi)$ and normal line $(x=\pi-e^\pi t, y=0, z=-t)$.
\end{enumerate}
}
%RESOLUTION
{
}


\newproblem{derimp-15}{gen}{}
%STATEMENT
{Derivar implicitamente para deducir una fórmula para las siguientes derivadas:
\begin{enumerate}
\item $(\arctg x)'$.
\item $(\arcsen x)'$.
\end{enumerate}
}
%SOLUTION
{\begin{enumerate}
\item $(\arctg x)'=\frac{1}{1+x^2}$.
\item $(\arcsen x)' = \frac{1}{\sqrt{1-x^2}}$.
\end{enumerate}
}
%RESOLUTION
{
}
