% AutHor: Alfredo Sánchez Alberca (asalber@ceu.es)

\newproblem{derpar-1}{gen}{}
%STATEMENT
{An object moves along the curve $y=\cos(2x+1)$, where $x=t^2+1$, and $t$ stands for time. Find the vertical and horizontal speed when
$t=2$.
}
%SOLUTION
{Horizontal speed: $\frac{dx}{dt} = 2t$; at instant $t=2$, $\frac{dx}{dt}(t=2)=4$.\\
Vertical speed: $\frac{dy}{dt}=-\sin(2t^2+3)4t$; at instant $t=2$, $\frac{dy}{dt}=-8\sin 11$.
}
%RESOLUTION
{
}


\newproblem{derpar-2}{gen}{}
%STATEMENT
{A point moves in the real plane following the trajectory
\[
\begin{cases}
x= \sin t,\\
y = t^2-1,
\end{cases}
\quad t\in \mathbb{R}.
\]
\begin{enumerate}
\item Compute the derivative of $y(x)$ (that is, $\dfrac{dy}{dx}$) at the points $t=0$ and $t=2$.
\item Compute the line tangent to the trajectory at the point $(0,-1)$.
\end{enumerate}
}
%SOLUTION
{\begin{enumerate}
\item $\frac{dy}{dx} = \frac{2t}{\cos t}$. $\frac{y}{dx}(t=0) = 0$ and $\frac{dy}{dx}(t=2) = 4/\cos 2$.
\item Tangent line: $y=-1$.
\end{enumerate}
}
%RESOLUTION
{
}


\newproblem*{derpar-3}{gen}{}
%STATEMENT
{Una partícula se mueve a lo largo de la curva
\[
\begin{cases}
x=2\sin t, \\
y=\sqrt{3}\cos t,
\end{cases}
\]
donde $x$ e $y$ están medidos en metros y el tiempo $t$ en
segundos.
\begin{enumerate}
\item  Hallar la ecuación de la recta tangente a la trayectoria en el punto (1,3/2).
\item  ¿Con qué velocidad se mueve la partícula respecto a las direcciones vertical y horizontal en dicho punto?
\end{enumerate}
}


\newproblem{derpar-4}{gen}{*}
%STATEMENT
{The coordinates of an object thrown horizontally are
\[
\begin{cases}
x=v_0t \\
y=-\frac{1}{2}gt^2
\end{cases}
\]
where $t\in \mathbb{R}^{+}$ is the time, $v_0$ is the horizontal velocity at time $t=0$, and $g=9.8$ m$^2$/s is the gravitational acceleration.
At what moment will the magnitudes of horizontal and vertical speeds be equal?
What should be the value of $v_0$ so that, at that moment (when horizontal and vertical speeds are equal), the object has traveled 100 m horizontally?
Find the equation of the tangent line to the trajectory at that time with the value of $v_0$ that you have computed previously.
}
%SOLUTION
{
Horizontal and vertical speeds are equal at time $ t=\frac{v_0}{9.8}$. To have traveled 100 m horizontally at that time, the initial speed should be $v_0 = 31.3$ m/s.\\
The equation of the tangent line to the trajectory at the point of biggest height is $y =-x+50.14$.
}
%RESOLUTION
{La velocidad horizontal es la derivada del espacio recorrido horizontalmente (componente $x$) con respecto al tiempo, es decir,
\[
\frac{dx}{dt} = \frac{d}{dt}(v_0t)=v_0.
\]
Del mismo modo, la velocidad vertical es la derivada del espacio recorrido verticalmente (componente $y$) en relación al tiempo,
\[
\frac{dy}{dt} = \frac{d}{dt}(-\frac{1}{2}gt^2)=-gt
\]
Para ver en qué instante ambas magnitudes serán iguales, las igualamos y resolvemos la ecuación:
\[
|\frac{dx}{dt}|=|\frac{dy}{dt}| \Leftrightarrow v_0 = gt \Leftrightarrow t=\frac{v_0}{g}=\frac{v_0}{9.8}.
\]

Para que en dicho instante el punto haya recorrido 100 m horizontalmente, debe cumplirse que $x(v_0/9.8)=100$, de lo que se deduce:
\[
x(v_0/9.8)=v_0\frac{v_0}{9.8} = \frac{v_0^2}{9.8}=100 \Leftrightarrow v_0^2 = 980 \Leftrightarrow v_0 = +\sqrt{980}= 31.3.
\]
Por tanto, el instance en cuestión es $t=v_0/9.8= 31.3/9.8 = 3.19$.

Por último, la ecuación de la recta tangente en dicho instante, para el valor de $v_0$ calculado es:
\[
y = y(3.19) + \frac{dy}{dx}(3.19) (x-x(3.19))
\]
Ya hemos visto que $x(3.19)=100$, y que en dicho instante la velocidad horizontal y vertical coinciden, de manera que
\[
\frac{dy}{dx}(3.19)=\frac{dy/dt}{dx/dt}=-1,
\]
de modo que sólo nos queda calcular el espacio vertical recorrido en dicho instante, que es
\[
y(3.19)=-\frac{1}{2}9.8\cdot 3.19^2= -49.86.
\]
Sustituyendo en la ecuación anterior llegamos a la recta tangente:
\[
y = -49.86-(x-100) \Leftrightarrow y=-x+50.14.
\]
}


\newproblem{derpar-5}{gen}{*}
%STATEMENT
{ Dada la función paramétrica
\[
\left(
x =\frac{(t-2)^2}{t^2+1},\, y=\frac{2t}{t^2+1}
\right)
\quad t\in \mathbb{R}.
\]
Calcular los valores máximos y mínimos de $x$ y de $y$.
¿En qué instante la tasa de crecimiento de $y$ coincide con la de $x$?
}
%SOLUTION
{$\frac{dx}{dt}=\frac{4t^2-6t-4}{(t^2+1)^2}$. Puntos críticos: $t=-1/2$ (máximo) y $t=2$ (mínimo).\\
$\frac{dy}{dt}=\frac{-2t^2+2}{(t^2+1)^2}$. Puntos críticos: $t=-1$ (mínimo) y $t=1$ (máximo).\\
$\frac{dx}{dt}=\frac{dy}{dt}$ en los instantes $t=\frac{1-\sqrt 5}{2}$ y $t=\frac{1+\sqrt 5}{2}$.
}
%RESOLUTION
{
}


\newproblem*{derpar-6}{gen}{*}
%STATEMENT
{Una mosca se mueve en un plano siguiendo la trayectoria
\[
\left\{
\begin{array}{lll}
x & = & \sin t
\; ,
\\
y & = & \cos t + t^2 - 1
\; .
\end{array}
\right.
\]
Se pide
\begin{enumerate}
\item Hallar la derivada de la función $y(x)$, es decir $dy/dx$,
en los puntos $t=0$ y $t=\pi/2$.
\item Hallar la ecuación de la recta tangente y normal a la trayectoria
en el punto $(x,y)=(0,0)$.
\end{enumerate}
}


\newproblem*{derpar-7}{gen}{*}
%STATEMENT
{Dadas las siguientes ecuaciones paramétricas:
\[
\left\{
\begin{array}{l}
x(t)=e^{at}t \\
y(t)=\ln t\cos (t-1)
\end{array}
\right.
\]
calcular la ecuación de la recta tangente a la gráfica de $y$ como función de $x$ en el punto que corresponde a $t=1$.
}


\newproblem*{derpar-8}{gen}{*}
%STATEMENT
{La cantidad de árboles en un ecosistema, $a$, depende del tiempo según la expresión:
\[
a(t)=100\ln(t^2+1)
\]
Y la cantidad de un determinado parásito de los árboles, $p$, que también depende del tiempo, viene dada por:
\[
p(t) = \sqrt[3]{{t^2  + 2}}
\]
Y se pide:
\begin{enumerate}
\item Calcular el número de parásitos cuando el número de árboles sea 500.
\item La derivada del número de parásitos con respecto al número de árboles cuando el número de parásitos sea 3.
\end{enumerate}
}


\newproblem*{derpar-9}{gen}{*}
%STATEMENT
{Supongamos un ecosistema en el que hay una especie ``presa", $p$, y otra ``depredador", $d$, y que la cantidad de individuos de una y otra dependen del tiempo, en años, según las siguientes expresiones ($t>0$):
\[
\renewcommand{\arraystretch}{2.2}
\begin{array}{*{20}c}
   {p(t) = \dfrac{{\ln (t^2  + 1)}}{{t + 1}}}  \\
   {d(t) = te^{ - 2t} }  \\
\end{array}
\]
\begin{enumerate}
\item Calcular el número de presas y depredadores para tiempos muy grandes.
\item Calcular la derivada del número de presas con respecto a los depredadores cuando $d=2/e^4$.
\end{enumerate}
}


\newproblem{derpar-10}{gen}{*}
%STATEMENT
{Un punto se mueve en el plano siguiendo una trayectoria
\[
\begin{cases}
x = \tg t,  \\
y = t^2-2t+3. \\
\end{cases}
\]

\begin{enumerate}
\item  Hallar $\frac{\partial y}{\partial x}$ en $t=0$.
\item  Hallar la tangente a la trayectoria en el punto $(0,3)$.
\end{enumerate}
}
%SOLUTION
{\begin{enumerate}
\item $\dfrac{\partial y}{\partial x}(t) = \frac{2t-2}{1+\tg^2t}$ y $\dfrac{\partial y}{\partial x}(0) = -2$.
\item $y = 3-2x$.
\end{enumerate}
}
%RESOLUTION
{Se trata de la ecuación de una trayectoria en coordenadas paramétricas.
\begin{enumerate}
\item  Aplicando la regla de la cadena se tiene que
\[
\dfrac{\partial y}{\partial t} = \dfrac{\partial y}{\partial x}\dfrac{\partial x}{\partial t},
\]
en consecuencia,
\[
\dfrac{\partial y}{\partial x}(t) = \frac{\partial y/\partial t}{\partial x/\partial t}(t)=\frac{2t-2}{1+\tg^2t}.
\]
En el punto $t=0$ tendremos
\[
\dfrac{\partial y}{\partial x}(0) = \frac{-2}{1+\tg^20} = -2.
\]

\item  La ecuación de la recta tangente a la trayectoria en el punto $(x(t_0),y(t_0))$ correspondiente al instante $t_0,$ viene dada por la expresión
\[
y-y(t_0) = \dfrac{\partial y}{\partial x}(t_0)(x-x(t_0)).
\]
Como el punto $(0,3)$ se alcanza precisamente en el instante $t=0$ tenemos que la ecuación de la recta tangente a la trayectoria en dicho instante es:
\[
y-y(0) = \dfrac{\partial y}{\partial x}(0)(x-x(0)),
\]
es decir,
\[
y-3 = -2(x-0),
\]
y simplificando obtenemos:
\[
y = 3-2x.
\]
\end{enumerate}
}



\newproblem{derpar-11}{gen}{}
%STATEMENT
{Compute the equations of the tangent and normal lines to the curve
$C$, at the point $P$, on each of the following exercises:
\begin{enumerate}
\item $C: y=x^2$, $P=(0,0)$
\item $C: \begin{cases}
x=2\cos t,\\
y=2\sin t,
\end{cases}
$ $0\leq t\leq 2\pi$, $P=(0,2)$
\item $C:x^2+y^2=1$, $P=(\sqrt{2}/2),\sqrt{2}/2)$
\item $C:(x-1)^2+y^2=4$, $P=(3,0)$
\item $C:x^2-y^2=1$, $P=(1,0)$
\item $C:\begin{cases}
x=e^t\cos t,\\
y=e^t\sin t,
\end{cases}
$, $t\in \mathbb{R}$, $P=(1,0)$
\end{enumerate}
}
%SOLUTION
{
\begin{enumerate}
\item Tangent line $y=0$ and normal line $x=0$.
\item Tangent line $y=2$ and normal line $x=0$.
\item Tangent line $y=-x+\sqrt{2}$ and normal line $y=x$.
\item Tangent line $x=3$ and normal line $y=0$.
\item Tangent line $x=1$ and normal line $y=0$.
\item Tangent line $y=x-1$ and normal line $y=-x+1$.
\end{enumerate}
}
%RESOLUTION
{
}


\newproblem{derpar-12}{gen}{}
%STATEMENT
{Find the equations of the tangent line and normal plane to the curve
\[
C:
\begin{cases}
x=\cos t \\
y=\sin t\\
z= t,
\end{cases}
\quad t\in \mathbb{R},
\]
at the point $P=(1,0,0)$.
}
%SOLUTION
{Tangent line: $(1,t,t)$. Normal plane: $y+z=0$.
}
%RESOLUTION
{
}


\newproblem{derpar-13}{gen}{}
%STATEMENT
{A trajectory pases through the point $(3,6,5)$ at the moment $t=0$ with speed $\mathbf{i}-\mathbf{k}$.
Compute the equations of the tangent line and the normal plane to the trajectory at that moment.
}
%SOLUTION
{Tangent line: $(3+t,6,5-t)$. Normal plane: $x-z+2=0$.
}
%RESOLUTION
{
}


\newproblem{derpar-14}{gen}{}
%STATEMENT
{An object moves along the following trajectory:
\[
\begin{cases}
x=e^t,\\
y=e^{-t},\\
z=\cos t,
\end{cases}
\quad t\in \mathbb{R};
\]
for negative values of $t$, and at $t=0$ it goes off along the tangent line (to the trajectory).
Where will the object be at time $t=3$?
}
%SOLUTION
{Recta tangente: $(1+t,1-t,1)$. Posición en el instante $t=3$: $(4,-2,1)$.
}
%RESOLUTION
{
}
