% Author: Alfredo Sánchez Alberca (asalber@ceu.es)

\newproblem{der-1}{gen}{}
%STATEMENT
{Estudiar si es derivable la función $f(x)=\sqrt[3]{x-1}$ en el punto $x=1$.
}
%SOLUTION
{La función no es derivable ya que $f'(1)=\lim_{h\rightarrow 0}\frac{f(1+h)-f(1)}{h} = \infty$.
}
%RESOLUTION
{Aplicando la definición de derivada, tenemos:
\begin{align*}
f'(1) = \lim_{h\rightarrow 0}\frac{f(1+h)-f(1)}{h} = \lim_{h\rightarrow 0}\frac{\sqrt[3]{1+h-1}-\sqrt[3]{1-1}}{h} = \lim_{h\rightarrow 0}\frac{\sqrt[3]{h}}{h} = \lim_{h\rightarrow 0} h^{-2/3} = \infty.
\end{align*}
Así pues, la función no es derivable en $x=1$.
}


\newproblem*{der-2}{gen}{}
%STATEMENT
{Comprobar que la función $f(x)=|x-1|$ es continua en $x=1$ pero no es derivable en dicho punto.
}


\newproblem*{der-3}{gen}{}
%STATEMENT
{Estudiar la derivabilidad de $f$ en los puntos $x=-1$, $x=2$ y $x=3$ siendo
\[ f(x)=
\begin{cases}
\log(-x) & \mbox{si $x<-1$,} \\
\sen \pi x & \mbox{si $x\in [-1,2]$,} \\
x/2 & \mbox{si $x\in (2,3)$,} \\
3/2 & \mbox{si $x\geq 3$.}
\end{cases}
\]
}


\newproblem*{der-4}{gen}{}
%STATEMENT
{Estudiar la derivabilidad y calcular la derivada, donde exista, de la función
\[ f(x)=
\begin{cases}
(x+3)^2 & \mbox{si $x\leq -2$,} \\
x^2-3 & \mbox{si $-2\leq x\leq -1$,} \\
0 & \mbox{si $-1< x <0$,} \\
x^2 & \mbox{si $0\leq x \leq 1$,} \\
\cos \dfrac{1}{x-1} & \mbox{si $1< x\leq 2$,} \\
2x-3 & \mbox{si $x>2$.}
\end{cases}
\]
}


\newproblem*{der-5}{gen}{}
%STATEMENT
{Probar que no es derivable en $x=0$ la siguiente función:
\[ f(x)=
\begin{cases}
e^x-1 & \mbox{si $x\geq 0$,}  \\
x^3 & \mbox{si $x<0$.}
\end{cases}
\]
}


\newproblem{der-6}{gen}{}
%STATEMENT
{Estudiar la derivabilidad de las siguientes funciones y hallar la función derivada correspondiente en los puntos donde exista.
\begin{enumerate}
\item  $f(x)=
\begin{cases}
1-x & \mbox{si $x\leq 0$,} \\
e^{-x} & \mbox{si $x>0$.}
\end{cases}
$
\item  $g(x)=2x+|x^2-2|$.
\end{enumerate}
}
%SOLUTION
{\begin{enumerate}
\item La función es derivable en $x=0$ ya que $f'^-(0)=f'^+(0) = 1$ y la derivada es
\[f'(x)=
\begin{cases}
-1 & \mbox{si $x\leq 0$,} \\
-e^{-x} & \mbox{si $x>0$.}
\end{cases}
\]
\item La función no es derivable en $x=-\sqrt{2}$ ya que $g'^-(-\sqrt{2})=2-2\sqrt 2$ y $g'^+(-\sqrt{2})=2+2\sqrt 2$, y tampoco es derivable en $x=\sqrt 2$ ya que $x=-\sqrt{2}$ ya que $g'^-(\sqrt{2})=2-2\sqrt 2$ y $g'^+(\sqrt{2})=2+2\sqrt 2$. En el resto de los puntos, la derivada vale
\[
g'(x)=
\begin{cases}
2+2x & \mbox{si $x< -\sqrt{2}$,} \\
2-2x & \mbox{si $-\sqrt 2 < x < \sqrt 2$,}\\
2+2x & \mbox{si $x > \sqrt 2$.}
\end{cases}
\]
\end{enumerate}
}
%RESOLUTION
{Para estudiar la derivabilidad primero vamos a expresar la función $|x^2-2|$ como una función a trozos. Para ello necesitamos saber en qué puntos la función $x^2-2$ es positiva, y en qué puntos es negativa. Si calculamos las raíces de esta función tenemos:
\[
|x^2-2| = 0 \Leftrightarrow x^2 = 2 \Leftrightarrow x = \pm \sqrt 2.
\]
Si estudiamos el signo en los intervalos definidos por las raíces, podemos comprobar fácilmente sin más que calcular la función en cualquier punto de los intervalos que $x^2-2$ es negativa en el intervalo $(-\sqrt 2, \sqrt 2)$ y positiva en el resto de su dominio.
Por tanto, podemos expresar el valor absoluto de la siguiente manera:
\[
|x^2-2| =
\begin{cases}
x^2-2 & \mbox{si $x< -\sqrt{2}$,} \\
-x^2+2 & \mbox{si $-\sqrt 2 \leq x \leq \sqrt 2$,}\\
x^2-2 & \mbox{si $x > \sqrt 2$.}
\end{cases}
\]
y entonces, la función original puede expresarse como:
\[
g(x) =
\begin{cases}
2x+x^2-2 & \mbox{si $x< -\sqrt{2}$,} \\
2x-x^2+2 & \mbox{si $-\sqrt 2 \leq x \leq \sqrt 2$,}\\
2x+x^2-2 & \mbox{si $x > \sqrt 2$.}
\end{cases}
\]
Ahora, si estudiamos la derivabilidad de cada una de estas funciones en los trozos correspondientes, vemos que ambas son polinomios y por tanto son derivables en sus dominios. Faltaría por estudiar la derivabilidad en los puntos donde cambia la definición de la función. Para ello hay que estudiar la derivada por la izquierda y por la derecha y ver si coinciden. En el punto $x=-\sqrt 2$ tenemos:

\begin{align*}
g'^-(-\sqrt{2}) &= 2-2\sqrt 2\\
g'^+(-\sqrt{2}) &= 2+2\sqrt 2
\end{align*}

Y como ámbas derivadas no coinciden la función no es derivable en $x=-\sqrt 2$. En $x=\sqrt 2 $ tenemos:

\begin{align*}
g'^-(\sqrt{2}) &= 2-2\sqrt 2\\
g'^+(\sqrt{2}) &= 2+2\sqrt 2
\end{align*}

Ambas derivadas no coinciden y tampoco es derivable en $x=\sqrt 2$.

Así pues, la derivada vale:
\[
g'(x)=
\begin{cases}
2+2x & \mbox{si $x< -\sqrt{2}$,} \\
2-2x & \mbox{si $-\sqrt 2 < x < \sqrt 2$,}\\
2+2x & \mbox{si $x > \sqrt 2$.}
\end{cases}
\]
}


\newproblem{der-7}{gen}{}
%STATEMENT
{Dada la función
\[ f(x)=
\begin{cases}
ax+\dfrac{1}{x} & \mbox{si $x\leq -1$,} \\
x^2+bx & \mbox{si $-1<x\leq 1$,} \\
\log (x^2) & \mbox{si $x>1$.}
\end{cases}
\]
donde $a$ y $b$ son constantes.
\begin{enumerate}
\item  ¿Existen algunos valores de las constantes para los que la función sea continua en todo su dominio? En caso afirmativo, indicar cuáles son esos valores, y en caso contrario, razonar la respuesta.
\item  ¿Existen algunos valores de las constantes para los que la función sea derivable en todo su dominio? En caso afirmativo, indicar cuáles son esos valores, y en caso contrario, razonar la respuesta.
\end{enumerate}
}
%SOLUTION
{\begin{enumerate}
\item La función es continua en todo el dominio si $a=-3$ y $b=-1$.
\item No existe ningún valor de $a$ y $b$ con los que la función sea derivable en todo el dominio.
\end{enumerate}
}
%RESOLUTION
{
}


\newproblem{der-8}{gen}{*}
%STATEMENT
{Dada la función
\[
f(x)=
\begin{cases}
\sen^2x & \mbox{si $x\leq 0$},  \\
ax^2+b &  \mbox{si $0<x\leq c$},  \\
\ln x &  \mbox{si $c<x$},
\end{cases}
\]
con $a$, $b$ y $c$ constantes, ¿existe algún valor de las constantes de manera que la función sea continua y derivable en todo su dominio?
}
%SOLUTION
{$a=\frac{1}{2e}, b=0, c=e^{1/2}.$
}
%RESOLUTION
{Estudiaremos primero la continuidad y luego la derivabilidad.

Las funciones $\sen 2x$, $ax^2+b$ y $\ln x$ son todas contínuas en sus dominios, por tanto, basta con estudiar los puntos donde cambia la definici\'{o}n de la funci\'{o}n.

En el punto $x=0$ tenemos:
\begin{align*}
\lim_{x\rightarrow 0^{-}}f(x) &= \lim_{x\rightarrow 0^{-}}\sen^2x=\sen^20=0, \\
\lim_{x\rightarrow 0^{+}}f(x) &= \lim_{x\rightarrow 0^{+}}ax^2+b=a0^2+b=b, \\
f(0) &= \sen^20=0.
\end{align*}

Luego la función será contínua en $x=0$ si y sólo si $b=0$.

En el punto $x=c$ tenemos:
\begin{align*}
\lim_{x\rightarrow c^{-}}f(x) &= \lim_{x\rightarrow c^{-}}ax^2+b=ac^2+b, \\
\lim_{x\rightarrow c^{+}}f(x) &= \lim_{x\rightarrow c^{+}}\ln x=\ln c, \\
f(c) &= ac^2+b.
\end{align*}

Luego la función será contínua en $x=c$ si y sólo si $ac^2+b=\ln c$.

Por consiguiente, para que la función sea contínua en todo su dominio deben cumplirse las dos ecuaciones siguientes:
\[
\begin{cases}
b=0 \\
ac^2+b = \ln c
\end{cases}
\]

Con la derivabilidad ocurre lo mismo pues las funciones $\sen 2x$ , $ax^2+b$ y $\ln x$ son derivables en su dominio y basta con estudiar la existencia de la derivada en los puntos donde cambia la definición de la función.

En el punto $x=0$ (imponemos $b=0$ pues de lo contrario la función no sería continua en este punto y tampoco derivable) tenemos:
\begin{align*}
\lim_{h\rightarrow 0^{-}}\frac{f(0+h)-f(0)}{h} &= \lim_{h\rightarrow 0^{-}}\frac{\sen^2(0+h)-\sen^20}{h} = \lim_{h\rightarrow 0^{-}}\frac{\sen^2h}{h}\stackrel{\text{L}^{\prime }\text{H\^{o}pital}}{=}\lim_{h\rightarrow 0^{-}}\frac{2\sen h\cos h}{1}=0, \\
\lim_{h\rightarrow 0^{+}}\frac{f(0+h)-f(0)}{h} &= \lim_{h\rightarrow 0^{-}}\frac{a(0+h)^2-\sen^20}{h}=\lim_{h\rightarrow 0^{-}}\frac{ah^2}h\stackrel{\text{L}^{\prime }\text{H\^{o}pital}}{=}\lim_{h\rightarrow 0^{-}}\frac{2ah}{1}=0.
\end{align*}

Luego la función es derivable en $x=0$ si y sólo si $b=0$.

En el punto $x=c$ tenemos:
\begin{align*}
\lim_{h\rightarrow 0^{-}}\frac{f(c+h)-f(c)}{h} &= \lim_{h\rightarrow 0^{-}}\frac{a(c+h)^2+b-(ac^2+b)}{h} = \lim_{h\rightarrow 0^{-}}\frac{ac^2+ah^2+2ach+b-ac^2-b}{h}= \\
&= \lim_{h\rightarrow 0^{-}}\frac{ah^2+2ach}{h} = \lim_{h\rightarrow 0^{-}}ah+2ac=2ac, \\
\lim_{h\rightarrow 0^{+}}\frac{f(0+h)-f(0)}{h} &= \lim_{h\rightarrow 0^{-}}\frac{\ln (c+h)-(ac^2+b)}{h}\stackrel{\text{L}^{\prime }\text{H\^{o}pital}}{=} \lim_{h\rightarrow 0^{-}}\frac{1/(c+h)}{1} = \frac{1}{c}.
\end{align*}

Observese que el último límite conduce a una indeterminación porque se debe cumplir $ac^2+b=\ln c,$ para que la función sea continua en dicho punto. Luego para que la función sea derivable en $x=c$, además de la condición de continuidad, se debe cumplir $2ac = \frac{1}{c}$.

Así pues, para que la función sea continua y derivable en todo su dominio deben cumplirse las tres ecuaciones siguientes:
\[
\begin{cases}
b=0 \\
ac^2+b = \ln c\\
2ac = \frac 1c
\end{cases}
\]

Resolviendo el sistema llegamos a:
\[
\renewcommand{\arraystretch}{2.5}
\begin{array}{c}
\displaystyle
a = \frac{1}{2c^2} \Longrightarrow \ln c = ac^2+b = \frac{1}{2c^2}c^2 = \frac{1}{2} \Longrightarrow c = e^{1/2}, \\
\displaystyle a = \frac{1}{2(e^{1/2})^2} = \frac{1}{2e},
\end{array}
\]

y los valores de las constantes que hacen que la función sea continua y derivable en todo su dominio son:
\begin{align*}
a &= \frac{1}{2e},\\
b &= 0, \\
c &= e^{1/2}.
\end{align*}
}


\newproblem{der-9}{gen}{}
%STATEMENT
{Hallar la función derivada de las siguientes funciones:
\begin{multicols}{2}
\begin{enumerate}
\item $f(x)=\tg(1+x)^3$.
\item $g(x)=\log_{3}(1+x)^2$.
\item $h(x)=\arcsen\dfrac{1+x}{1-x}$.
\item $i(x)= \arctg(\sqrt{x^2+1})$.
\end{enumerate}
\end{multicols}
}
%SOLUTION
{\begin{enumerate}
\item $\frac{df}{dx} = (1 + \tg^2((1+x)^3))3(1+x)^2$ y $df = (1 + \tg^2((1+x)^3))3(1+x)^2 dx$.
\item $\frac{dg}{dx} = \frac{2(1+x)\log_3e}{(1+x)^2}$ y $dg = \frac{2(1+x)\log_3e}{(1+x)^2} dx$.
\item $\frac{dh}{dx} = \frac{1}{\sqrt{1-\left(\frac{1+x}{1-x}\right)^2}}\frac{2}{(1-x)^2}$ y $dh = \frac{1}{\sqrt{1-\left(\frac{1+x}{1-x}\right)^2}}\frac{2}{(1-x)^2} dx$.
\item $\frac{di}{dx} = \frac{1}{x^2+2}\frac{x}{\sqrt{x^2+1}}$ y $di = \frac{1}{x^2+2}\frac{x}{\sqrt{x^2+1}} dx$.
\end{enumerate}
}
%RESOLUTION
{
}


\newproblem{der-10}{gen}{}
%STATEMENT
{Find an equation of the tangent and normal lines to the curves given below at the given point $x_0$.
\begin{multicols}{2}
\begin{enumerate}
\item  $y=x^{\sin x},\quad x_{0}=\pi/2$.
%\item  $y=(3-x^2)^4\sqrt[3]{5x-4},\quad x_{0}=1$.
\item  $y=\log \sqrt{\dfrac{1+x}{1-x}}, \quad x_{0}=0$.
\end{enumerate}
\end{multicols}
}
%SOLUTION
{\begin{enumerate}
\item Tangent: $y-\frac{\pi}{2} = x-\frac{\pi}{2}$. Normal: $y-\frac{\pi}{2} = -x+\frac{\pi}{2}$.
%\item Tangente: $y - 2^4 = -\frac{112}{3}(x-1)$. Normal: $y - 2^4 = \frac{3}{112}(x-1)$.
\item Tangent: $y = x$. Normal: $y = -x$.
\end{enumerate}
}
%RESOLUTION
{
}


\newproblem*{der-11}{gen}{}
%STATEMENT
{Determinar el ángulo formado por las curvas $y=x^4+1$ e $y=5x^2-3$ en el punto $x_{0}=1$.

\noindent \textbf{Nota}: El ángulo que forman dos curvas es el ángulo
que forman sus tangentes.
}


\newproblem{der-12}{gen}{*}
%STATEMENT
{Dadas las funciones $f(x)=\log \left(\dfrac{x^2}{2}\right)$ y $g(x)=x^3+2$, ¿existe algún valor de $x$ en el que la recta normal a $f$ y
la recta tangente a $g$ en dicho punto sean paralelas? }
%SOLUTION
{$x=-1/6$.
}
%RESOLUTION
{
}


\newproblem{der-13}{gen}{}
%STATEMENT
{Hallar la expresión de la derivada $n$-ésima de las siguientes funciones:
\begin{multicols}{2}
\begin{enumerate}
\item  $f(x)=a^x\log a$.
\item  $g(x)=\dfrac{\sen x+\cos x}{2}$.
\item  $h(x)=\dfrac{9x^2-2x-25}{x^3-2x^2-5x+6}$.
\item  $j(x)=\dfrac{1}{\sqrt{1+x}}$.
\end{enumerate}
\end{multicols}
}
%SOLUTION
{\begin{enumerate}
\item $f^{(n} = (\log a)^{n+1} a^x$.
\item $g^{(n}=
\begin{cases}
\frac{\sen x + \cos x}{2} & \mbox{si $x=4k$,}\\
\frac{\cos x - \sen x}{2} & \mbox{si $x=4k+1$,}\\
\frac{-\sen x - \cos x}{2} & \mbox{si $x=4k+2$,}\\
\frac{-\cos x + \sen x}{2} & \mbox{si $x=4k+3$.}\\
\end{cases}$
\item Descomponiendo en fracciones simples, $h^{(n}(x) = \frac{3(-1)^n n!}{(x-1)^{n+1}} + \frac{(-1)^n n!}{(x+2)^{n+1}} + \frac{5(-1)^n n!}{(x-3)^{n+1}}$.
\item $j^{(n}(x) = \frac{(-1)^n \prod_{i=1}^{n}2i-1}{2^n}(1+x)^{-\frac{2n+1}{2}}$.
\end{enumerate}
}
%RESOLUTION
{
}


\newproblem{der-14}{gen}{*}
%STATEMENT
{Dada la función
\[f(x)=
\begin{cases}
ax^2+bx, & \mbox{si $x<-\pi$;} \\
\cos x, & \mbox{si $-\pi\leq x\leq \pi$;} \\
cx^2+dx, & \mbox{si $x>\pi$.} \\
\end{cases}
\]
Calcular $a$, $b$, $c$ y $d$ para que la función sea continua y derivable en todo $\mathbb{R}$.
}
%SOLUTION
{$a = \frac{1}{\pi^{2}}$, $b=\frac{2}{\pi}$, $c = \frac{1}{\pi^{2}}$ y $d=\frac{-2}{\pi}$.
}
%RESOLUTION
{Estudiamos primero la continuidad. Los distintos trozos de la función están formados por funciones polinómicas y la función $\cos x$ que están definidas en todo su dominio, de manera que los únicos posibles puntos de discuntinuidad son los puntos donde cambia la definición de la función. Calculamos los límites laterales en dichos puntos: En el punto $x=-\pi$ tenemos
\begin{align*}
\lim_{x\rightarrow -\pi^{-}} f(x) &=  \lim_{x\rightarrow -\pi^{-}} ax^{2}+bx = a\pi^{2}-b\pi,\\
\lim_{x\rightarrow -\pi^{+}} f(x) &=  \lim_{x\rightarrow -\pi^{+}} \cos x = \cos -\pi = -1,
\end{align*}
de modo que para que la función sea continua en este punto debe cumplirse
\begin{equation}
a\pi^{2}-b\pi = -1
\label{e:1}
\end{equation}
Y en el punto $x=\pi$ tenemos
\begin{align*}
\lim_{x\rightarrow \pi^{-}} f(x) &=  \lim_{x\rightarrow \pi^{-}} \cos x = \cos\pi = -1,\\
\lim_{x\rightarrow \pi^{+}} f(x) &=  \lim_{x\rightarrow \pi^{+}} cx^{2}+dx =  c\pi^{2}+d\pi,
\end{align*}
de modo que para que la función sea continua en este punto debe cumplirse
\begin{equation}
c\pi^{2}+d\pi = -1
\label{e:2}
\end{equation}

En cuanto a la derivabilidad, calculamos primero las derivadas de las funciones de cada uno de los trozos:
\[f'(x)=
\left\{%
\begin{array}{ll}
  2ax+b, & \hbox{si $x<-\pi$;} \\
  -\sen x, & \hbox{si $-\pi< x< \pi$;} \\
  2cx+d, & \hbox{si $x>\pi$.} \\
\end{array}%
\right.
\]
Al igual que antes, las derivadas de las funciones de cada uno de los trozos existen en sus respectivos dominios por lo que faltaría por ver si existe la derivada en los puntos donde cambia la definición de la función. Calculamos las derivadas laterales en dichos puntos: En el punto $x=-\pi$ tenemos
\begin{align*}
\lim_{x\rightarrow -\pi^{-}} f'(x) &=  \lim_{x\rightarrow -\pi^{-}} 2ax+b = -2a\pi+b,\\
\lim_{x\rightarrow -\pi^{+}} f'(x) &=  \lim_{x\rightarrow -\pi^{+}} -\sen x = -\sen -\pi = 0,
\end{align*}
de modo que para que la función sea derivable en este punto, además de la condición~\ref{e:1} debe cumplirse la condición
\begin{equation}
-2a\pi+b = 0
\label{e:3}
\end{equation}
Y en el punto $x=\pi$ tenemos
\begin{align*}
\lim_{x\rightarrow \pi^{-}} f'(x) &=  \lim_{x\rightarrow -\pi^{-}} -\sen x = -\sen \pi = 0,\\
\lim_{x\rightarrow \pi^{+}} f'(x) &=  \lim_{x\rightarrow -\pi^{+}} 2cx+d = 2c\pi+d,
\end{align*}
de modo que para que la función sea derivable en este punto, además de la condición~\ref{e:2} debe cumplirse la condición
\begin{equation}
2c\pi+d = 0
\label{e:4}
\end{equation}

Así pues, para que $f(x)$ sea continua y derivable en el punto $x=-\pi$ deben cumplirse las ecuaciones~\ref{e:1} y \ref{e:3}. Resolviendo el sistema que forman llegamos a que
\[
a = \frac{1}{\pi^{2}} \qquad \mbox{y} \qquad b=\frac{2}{\pi}.
\]
Y para que sea continua y derivable en el punto $x=\pi$ deben cumplirse las ecuaciones~\ref{e:2} y \ref{e:4}. Resolviendo el sistema que forman llegamos a que
\[
c = \frac{1}{\pi^{2}} \qquad \mbox{y} \qquad d=\frac{-2}{\pi}.
\]
}


\newproblem{der-15}{gen}{*}
%STATEMENT
{Se  considera la función $f:\mathbb{R}\to\mathbb{R}$ definida por:
\[\renewcommand{\arraystretch}{2}
f(x) =
\begin{cases}
\dfrac{x^2  + 1}{x - 1} & \mbox{si $x \leq 0$,} \\
\dfrac{ax + b}{x^2  + 2x + 1} & \mbox{si $x>0$.} \\
\end{cases}
\]
siendo $a$ y $b$ $\in \mathbb{R}$.
\begin{enumerate}
\item Hallar $a$ y $b$ para que la función $f$ sea continua en todo $\mathbb{R}$ y su derivada se anule en $x=2$.
Con los valores de $a$ y $b$ obtenidos en el apartado anterior:
\item Estudiar la derivabilidad de la función $f$.
\item Hallar las asíntotas de $f$.
\end{enumerate}
}
%SOLUTION
{\begin{enumerate}
\item $a=2$ y $b=-1$.
\item La función es derivable en  todo $\mathbb{R}$ excepto en el 0.
\item $y=0$ es asíntota horizontal por la derecha y  $y=x+1$ es una asíntota oblicua.
\end{enumerate}
}
%RESOLUTION
{\begin{enumerate}
\item Estudiemos primero la continuidad de la función. Puesto que se trata de una función definida por tramos, primero estudiamos cada tramo por separado. En el primer tramo $x< 0$ la función vale $\frac{x^2+1}{x-1}$ que es una función racional, y por tanto,  está definida y es continua en todos los puntos excepto en aquellos que anulen en denominador. Pero el único punto que anula el denominador es $x=1$ que se sale del tramo de definición de la función, por lo que en el primer tramo la función es continua. En el segundo tramo $x>0$ la función vale $\frac{ax + b}
{x^2  + 2x + 1}=\frac{ax+b}{(x+1)^2}$ que también es una función racional. El único punto que anula el denominador es $x=-1$, pero al igual que antes, se sale del intervalo de definición de la función, por lo que también es continua en su tramo. Por último queda estudiar la continuidad en $x=0$, que es donde cambia la definición de la función, por lo que procedemos a calcular los límites laterales:
\begin{align*}
\lim_{x\rightarrow 0^-}f(x)&=\lim_{x\rightarrow 0^-}\frac{x^2+1}{x-1}=\frac{0^2+1}{0-1}=-1,\\
\lim_{x\rightarrow 0^+}f(x)&=\lim_{x\rightarrow 0^+}\frac{ax+b}{x^2+2x+1}=\frac{a\cdot 0+b}{0^2+2\cdot 0+1}=b,
\end{align*}
de donde se deduce que para que exista el límite debe ser $b=-1$. En este caso, el límite coincidiría con el valor de la función $f(0)=-1$, por lo que la función sería continua en todo $\mathbb{R}$.

Estudiemos ahora la derivada en $x=2$. Puesto que el punto pertenece al segundo tramo de la función, derivamos este tramo
\begin{align*}
\frac{d}{dx}\left(\frac{ax+b}{x^2  + 2x + 1}\right) &=\frac{\frac{d}{dx}(ax+b)(x^2+2x+1)-(ax+b)\frac{d}{dx}(x^2+2x+1)}{(x^2+2x+1)^2}=\\
&=\frac{a(x^2+2x+1)-(ax+b)(2x+2)}{(x^2+2x+1)^2}
\end{align*}
Sustituyendo en $x=2$ y $b=-1$ tenemos
\[f'(2)=\frac{a(2^2+2\cdot 2+1)-(2a-1)(2\cdot2+2)}{(2^2+2\cdot2+1)^2}=\frac{-3a+6}{81}=0 \Leftrightarrow -3a+6=0 \Leftrightarrow a=2.\]

\item Antes de estudiar la derivabilidad, sustituimos $a$ y $b$ por los valores obtenidos con lo que tenemos la función
\[
\renewcommand{\arraystretch}{2}
\begin{cases}
\dfrac{x^2  + 1}{x - 1} & \mbox{si $x \leq 0$,} \\
\dfrac{2x-1}{x^2  + 2x + 1} & \mbox{si $x>0$.} \\
\end{cases}
\]
Para estudiar la derivabilidad, derivamos cada tramo por separado. La derivada del segundo tramo la calculamos en el apartado anterior, así que falta calcular la del primer tramo que es inmediata
\begin{align*}
\frac{d}{dx}\left(\frac{x^2+1}{x-1}\right) &=\frac{\frac{d}{dx}(x^2+1)(x-1)-(x^2+1)\frac{d}{dx}(x-1)}{(x-1)^2}=\\
&=\frac{2x(x-1)-(x^2+1)1}{(x-1)^2}=\frac{x^2-2x-1}{(x-1)^2}
\end{align*}
Así pues, a falta de estudiar la derivabilidad en el punto $x=0$, tenemos que $f$ es derivable en cada un de los tramos y sus derivadas valen
\[\renewcommand{\arraystretch}{2}
f'(x)=\left\{%
\begin{array}{ll}
\dfrac{x^2-2x-1}{(x-1)^2}, & \hbox{si $x<0$;} \\
\dfrac{-2x^2+2x+4}{(x^2+2x+1)^2}, & \hbox{si $x>0$.} \\
\end{array}%
\right.
\]
Por último estudiamos la derivabilidad en $x=0$ mirando la derivada por la izquierda y por la derecha:
\begin{align*}
f'^-(0)=\lim_{x\rightarrow 0^-}f'(x)&=\lim_{x\rightarrow 0^-}\frac{x^2-2x-1}{(x-1)^2}=\frac{0^2-2\cdot 0-1}{(0-1)^2}=-1,\\
f'^+(0)=\lim_{x\rightarrow 0^+}f'(x)&=\lim_{x\rightarrow 0^+}\frac{-2x^2+2x+4}{(x^2+2x+1)^2}=\frac{-2\cdot 0^2+2\cdot 0+4}{(0^2+2\cdot 0+1)^2}=4,
\end{align*}
y como no coinciden, la función es derivable en en todo $\mathbb{R}$ excepto en el 0, y su derivada es la anterior.

\begin{itemize}
\item Asíntotas Verticales. Puesto que la función está definida en todo $\mathbb{R}$ y además es continua, no existen puntos en los que la función tienda a $\pm\infty$, de modo que no hay asíntotas verticales.

\item Asíntotas Horizontales. Para las asíntotas horizontales estudiamos la tendencia de $f$ en $\pm\infty$:
\begin{align*}
\lim_{x\rightarrow -\infty}f(x)&=\lim_{x\rightarrow -\infty}\frac{x^2+1}{x-1}\stackrel{(1)}{=}\lim_{x\rightarrow -\infty}\frac{2x}{1}=-\infty,\\
\lim_{x\rightarrow \infty}f(x)&=\lim_{x\rightarrow \infty}\frac{2x-1}{x^2+2x+1}\stackrel{(1)}{=}\lim_{x\rightarrow \infty}\frac{2}{2x+2}=0,
\end{align*}
\begin{quote}
\footnotesize
(1) Indeterminación del tipo $0/0$. Aplicamos la regla de L'Hôpital.
\end{quote}
Así pues, la recta $y=0$ es asíntota horizontal por la derecha. Por la izquierda no hay asíntota horizontal.

\item Asíntotas oblicuas. Estudiamos la existencia de asíntotas oblicuas sólo por la izquierda porque por la derecha no puede existir asíntota oblicua al haber asíntota horizontal.
\begin{align*}
\lim_{x\rightarrow -\infty}\frac{f(x)}{x}&=\lim_{x\rightarrow -\infty}\frac{x^2+1}{(x-1)x}=\lim_{x\rightarrow -\infty}\frac{x^2+1}{x^2-x}\stackrel{(1)}{=}\lim_{x\rightarrow -\infty}\frac{2x}{2x-1}\stackrel{(1)}{=}\lim_{x\rightarrow -\infty}\frac{2}{2}=1.
\end{align*}
\begin{quote}
\footnotesize
(1) Indeterminación del tipo $0/0$. Aplicamos la regla de L'Hôpital.
\end{quote}
Así pues, existe asíntota oblicua con pendiente 1. El término independiente nos lo da el siguiente límite
\begin{align*}
\lim_{x\rightarrow -\infty}f(x)-x&=\lim_{x\rightarrow -\infty}\frac{x^2+1}{x-1}-x=\lim_{x\rightarrow -\infty}\frac{x^2+1-x^2+x}{x-1}=\\
&=\lim_{x\rightarrow -\infty}\frac{x+1}{x-1}\stackrel{(1)}{=}\lim_{x\rightarrow -\infty}\frac{1}{1}=1,
\end{align*}
\begin{quote}
\footnotesize
(1) Indeterminación del tipo $0/0$. Aplicamos la regla de L'Hôpital.
\end{quote}
de manera que la ecuación de la asíntota oblicua es $y=x+1$.
\end{itemize}
\end{enumerate}
}


\newproblem{der-16}{gen}{*}
%STATEMENT
{Dada la función:
\[
\renewcommand{\arraystretch}{2}
f(x)=
\begin{cases}
\dfrac{1}{1 - 2^\frac{x}{1-x}} & \mbox{si $x\ne 1$,} \\
0 & \mbox{si $x = 1$.} \\
\end{cases}
\]
\begin{enumerate}
\item Estudiar su continuidad en $x=1$.
\item Mediante la definición de derivada de una función en un punto, calcular tanto la derivada por la derecha como la derivada por la izquierda en $x=1$.
\end{enumerate}
}
%SOLUTION
{\begin{enumerate}
\item En $x=1$ la función presenta una discontinuidad de 1ª especie de salto finito.
\item La dervidad por la izquierda en $x=1$ vale $f'^-(1)=0$ y no existe la derivada por la derecha en dicho punto.
\end{enumerate}
}
%RESOLUTION
{
\begin{enumerate}
\item Para que la función sea continua en $x=1$ debe cumplirse que $\lim_{x\rightarrow 1}f(x)=f(1)$. En primer lugar, la función está bien definida en $x=1$ y $f(1)=0$. Veamos ahora los límites laterales:
\begin{align*}
\lim_{x\rightarrow 1^-} f(x) &= \lim_{x\rightarrow 1^-} \frac{1}{1-2^{\frac{x}{1-x}}} = \frac{1}{1-2^{+\infty}}= \frac{1}{1-\infty}=\frac{1}{-\infty}=0,\\
\lim_{x\rightarrow 1^+} f(x) &= \lim_{x\rightarrow 1^+} \frac{1}{1-2^{\frac{x}{1-x}}} = \frac{1}{1-2^{-\infty}}= \frac{1}{1-0}=\frac{1}{1}=1.\\
\end{align*}
Así pues, como $\lim_{x\rightarrow 1^-}\neq \lim_{x\rightarrow 1^+}$, no existe el límite y la función presenta una discontinuidad de 1ª especie de salto finito.

\item Puesto que la función no es continua en $x=1$, tampoco será derivable. No obstante, pueden existir las derivadas laterales. Pasamos a calcularlas mediante la definición de derivada:
\begin{align*}
f'^-(1)&= \lim_{h\rightarrow 0^-}\frac{f(1+h)-f(1)}{h}= \lim_{h\rightarrow 0^-}\frac{\frac{1}{1-2^{\frac{1+h}{1-(1+h)}}}-0}{h} =  \lim_{h\rightarrow 0^-}\frac{1/h}{1-2^{\frac{1+h}{-h}}}=\\
&= \frac{1/0^-}{1-2^{\frac{1+0}{0^+}}} =\frac{-\infty}{1-2^{+\infty}} =\frac{-\infty}{-\infty},
\end{align*}
que es una indeterminación. Aplicando la regla de L'Hôpital tenemos
\begin{align*}
f'^-(1)&= \lim_{h\rightarrow 0^-}\frac{(1/h)'}{\left(1-2^{\frac{1+h}{-h}}\right)'} = \lim_{h\rightarrow 0^-}\frac{-1/h^2}{-2^{\frac{1+h}{-h}}\log 2 \left(\frac{-h+1+h}{h^2}\right)} = \lim_{h\rightarrow 0^-}\frac{-1/h^2}{-2^{\frac{1+h}{-h}}\log 2 \left(\frac{1}{h^2}\right)} \\
&= \lim_{h\rightarrow 0^-}\frac{1}{2^{\frac{1+h}{-h}}\log 2 } =\frac{1}{2^{+\infty}\log 2}=\frac{1}{+\infty}=0,\\
f'^+(1)&= \lim_{h\rightarrow 0^+}\frac{f(1+h)-f(1)}{h}=
\lim_{h\rightarrow 0^+}\frac{\frac{1}{1-2^{\frac{1+h}{1-(1+h)}}}-0}{h} =
\lim_{h\rightarrow 0^+}\frac{\frac{1}{1-2^{\frac{1+h}{-h}}}-0}{h} =
\frac{\frac{1}{1-2^{\frac{1+0}{0^-}}}}{0^+} = \\
&= \frac{\frac{1}{1-2^{-\infty}}}{0^+} =
\frac{\frac{1}{1-0}}{0^+}=\frac{1}{0^+}=+\infty.
\end{align*}
En consecuencia, existe derivada por la izquierda pero no por la derecha.
\end{enumerate}
}


\newproblem*{der-17}{gen}{}
%STATEMENT
{Calcular la derivada de las siguientes funciones:
\begin{multicols}{2}
\begin{enumerate}
\item $y=x^3+2x^2-3x+8$
\item $y=\dfrac{1}{x^2}-2x^{-1}+\sqrt{3x}$
\item $y=\dfrac{2x+1}{2x-1}$
\item $y=8^{3x^2-1}$
\item $y=\log \dfrac{x-1}{x+1}$
\item $y=\sqrt{\dfrac{1-x}{1+x}}$
\item $y=e^{2x}\log x^2$
\item $y=\log_7(x^2+2x+1)$
\item $y=\sqrt[4]{x^2-3x}$
\item $y=\sqrt{x-1}-\sqrt{x+1}$
\item $y=(\log x+\sqrt{x})^3$
\item $y=\dfrac{\log x}{e^x}$
\item $y=\log \sqrt{\dfrac{x}{x-2}}$
\item $y=\sen^2(x^2+3x)$
\item $y=\cos(\log x^2)$
\item $y=\tg(x^2-2)$
\item $y=2^{\log \cos x}$
\item $y=\log\left(\cos\dfrac{x^2}{2}\right)$
\item $y=\dfrac{1-\cos x}{1+\cos x}$
\item $y=\log\sqrt{\dfrac{1-\sen 2x}{1+\sen 2x}}$
\item $y=\dfrac{1}{2}\log \tg \dfrac{x}{2}-\dfrac{\cos x}{2\sen^2 x}$
\item $y=\arcsen \dfrac{x^3}{2}$
\item $y=\arcsen(\sen x^2)+\arcsen(\cos x^2)$
\item $y=x^x$
\item $y=(\sen x)^{\cos x}$
\end{enumerate}
\end{multicols}
}


\newproblem*{der-18}{gen}{}
%STATEMENT
{Hallar las derivadas sucesivas hasta la quinta de las siguientes funciones:
\begin{multicols}{2}
\begin{enumerate}
\item $y=3x^2-2x+5$
\item $y=\dfrac{1}{x}.$
\item $y=\log(x+2)$
\item $y=\dfrac{x-1}{x+3}$
\item $y=\dfrac{2x}{x^2-1}$
\item $e^{2\cos x}$
\end{enumerate}
\end{multicols}
}


\newproblem*{der-19}{gen}{*}
%STATEMENT
{Sea la función $f(x)=(x^{2}-x)3^{x/2}.$ Calcular su derivada n-ésima. Aplicar el resultado anterior para calcular la derivada de orden $100$.
}


\newproblem*{der-20}{gen}{*}
%STATEMENT
{Sea
\[
f(x)=
\begin{cases}
ae^{-x^2} & \mbox{si $x<1$} \\
b\log (1/x)+1 & \mbox{si $x\geq 1$}
\end{cases}
\]
Calcular $a$ y $b$ para que la función sea continua y derivable en en cualquier valor de $x$.
}


\newproblem{der-21}{gen}{*}
%STATEMENT
{Calcular la derivada $n$-ésima de la siguiente función
\[
f(x)= \frac{x^2+2x+3}{x+2}.
\]
Apoyándose en el cálculo anterior, dar la expresión de la derivada de orden 20 de $f$.
}
%SOLUTION
{$f^{(n}(x) = (-1)^n 3n!(x+2)^{-n+1}-$ y $f^{(20}(x) = 3\cdot 20!(x+2)^{-21}$.
}
%RESOLUTION
{
}


\newproblem{der-22}{gen}{*}
%STATEMENT
{Analizar la continuidad y la derivabilidad de la siguiente función
\[
f(x)=
\begin{cases}
-1-\dfrac{x^{2}}{\left| x\right| } & \mbox{si $x<0$}, \\
-2+e^{-x} & \mbox{si $x\geq 0$}.
\end{cases}
\]

¿Qué sucede si cambiamos en el STATEMENT $e^{-x}$ por $e^{x}$?
}
%SOLUTION
{La función es continua en todo $\mathbb{R}$ y derivable en $\mathbb{R}-\{0\}$.
Si se cambia $e^{-x}$ por $e^{x}$ la función pasa a ser derivable en todo $\mathbb{R}$.
}
%RESOLUTION
{Antes de estudiar la continuidad nos interesa eliminar el valor absoluto que aparece en la primera rama de deficinición de la función. Si observamos la definición de la función, la primera rama de definición es para los negativos ($x<0$); en consecuencia, como el argumento del valor absoluto es precisamente $x$, lo que hará el valor absoluto será cambiar de signo el valor de $x$. Por lo tanto, podemos sustituir $\left|x\right|$ por $-x$ en la primera rama de definición y de esta forma la función quedaría definida como sigue:
\[
f(x)=
\begin{cases}
-1+\dfrac{x^{2}}{x} & \mbox{si $x<0$}, \\
-2+e^{-x} & \mbox{si $x\geq 0$}.
\end{cases}
\]

Por otro lado, el cociente $\dfrac{x^{2}}{x}$ de la primera rama también puede simplificarse ya que en dicha rama no está incluido el 0 que sería el único valor que anularía el cociente. Es decir, tras simplificar, la función con la que tenemos que trabajar es
\[
f(x)=
\begin{cases}
-1+x & \mbox{si $x<0$}, \\
-2+e^{-x} & \mbox{si $x\geq 0$}.
\end{cases}
\]

Para estudiar la continuidad, comprobamos que la primera rama contiene un polinomio que es continuo en todo su dominio, mientras que la segunda rama contiene una función exponencial compuesta con una suma, y también es continua en todo su dominio. En consecuencia, el único punto en el que queda por estudiar la continuidad es en el cambia la definición de la función, es decir, en el 0.

Para estudiar la continuidad en el 0, calculamos los límites por la izquierda y por la derecha:
\begin{align*}
\lim_{x\rightarrow 0^{-}}f(x) &= \lim_{x\rightarrow 0^{-}}-1+x=-1, \\
\lim_{x\rightarrow 0^{+}}f(x) &= \lim_{x\rightarrow 0^{+}}-2+e^{-x}=-2+1=-1.
\end{align*}
Como ambos límites coinciden y además $f(0)=-2+e^{-0}=-2+1=-1,$ también coincide con el valor del límite, concluimos que la función es continua en el 0 y por consiguiente en todo $\mathbb{R}$.

Para estudiar la derivabilidad, comprobamos que el polinomio de la primera rama es derivable en todo su dominio y su derivada vale 1. Del mismo modo la función de la segunda rama también es derivable en todo su dominio y su derivada vale $-e^{-x}$. De nuevo, el único punto que queda por estudiar es el 0.

Para estudiar la derivabilidad en el 0 tomamos la función derivada
\[
f^{\prime }(x)=
\begin{cases}
1 & \mbox{si $x<0$}, \\
-e^{-x} & \mbox{si $x>0$},
\end{cases}
\]
y calculamos los límites laterales en el 0:
\begin{align*}
\lim_{x\rightarrow 0^{-}}f^{\prime }(x) &= \lim_{x\rightarrow 0^{-}}1=1, \\
\lim_{x\rightarrow 0^{+}}f^{\prime }(x) &= \lim_{x\rightarrow 0^{+}}-e^{-x}=-1.
\end{align*}
Como ambos límites son distintos, no existe derivada en el 0 y la función es derivable en $\mathbb{R}-\{0\}$.

Por último, si cambiamos $e^{-x}$ por $e^{x}$ en la definición de la función, entonces la continuidad seguiría igual pues el límite por la derecha en el 0 no cambiaría
\[
\lim_{x\rightarrow 0^{+}}f(x) = \lim_{x\rightarrow 0^{+}}-2+e^{x}=-2+1=-1.
\]
Sin embargo, la función derivada sí cambiaría:
\[
f^{\prime }(x)=
\begin{cases}
1 & \mbox{si $x<0$}, \\
e^{x} & \mbox{si $x>0$}.
\end{cases}
\]
Y ahora, al calcular el límite por la derecha en el 0 tendríamos
\[
\lim_{x\rightarrow 0^{+}}f^{\prime }(x) = \lim_{x\rightarrow 0^{+}}e^{x}=1.
\]
que coincide con el límite por la izquierda. Así pues, la función sería derivable en el 0 y, por lo tanto, en todo $\mathbb{R}$.
Además, la función derivada sería
\[
f^{\prime }(x)=
\begin{cases}
1 & \mbox{si $x<0$}, \\
e^{x} & \mbox{si $x\geq 0$}.
\end{cases}
\]
}


\newproblem{der-23}{gen}{}
%STATEMENT
{Air is being pumped into a spherical balloon of radius 10cm so that the radius increases at a rate of 2 cm/s.
How fast will the volume of the balloon increase?\\
Remark: The volume of a sphere is given by $V=\frac{4}{3}\pi r^3$.
}
%SOLUTION
{$800\pi$ cm$^3$/s.
}
%RESOLUTION
{
}


\newproblem{der-24}{gen}{}
%STATEMENT
{En muchos vertebrados existe una relación entre la longitud del cráneo y la longitud de la espina dorsal que puede expresarse mediante la ecuación
\[
C(x) = a E(x)^b
\]
donde $a$ es una constante de proporcionalidad y $b$ es otra constante que suele estar entre 0 y 1.
Esta ecuación se conoce como \emph{ecuación alométrica}.
¿Cómo se relaciona la tasa de crecimiento de la espina dorsal con la del cráneo?
¿Para qué valores de $b$ es la función $C$ creciente, pero de forma que la relación $C/E$ disminuye al aumentar $E$?
¿En qué estado de desarrollo tienen los vertebrados cráneos mayores en relación con la longitud de sus cuerpos?
}
%SOLUTION
{$800\pi$ cm$^3$/s.
}
%RESOLUTION
{
}


\newproblem{der-25}{gen}{}
%STATEMENT
{A car is moving on a straight line direction, with position given by the following function:
\[
e(t) = 4t^3 -2t +1.
\]
Find the speed and acceleration of the car.\\
Remark: The acceleration is the variation rate of the instant velocity.
}
%SOLUTION
{speed $v(t)=12t^2-2$ and acceleration $a(t)=24t$.
}
%RESOLUTION
{
}


\newproblem{der-26}{gen}{}
%STATEMENT
{An object is thrown verticall upwards.
Assuming there is no air friction, the object will travel a distance given by the following equation:
\[
e(t) =v_0t-\frac{1}{2}gt^2
\]
where $v_0$ is the initial velocity (at which the object is thrown), $g=9.81$ m/s$^2$ is the gravitational Earth constant, and $t$ is the time lapsed since the object was thrown.
\begin{enumerate}
\item Compute the speed and acceleration of the object at any time.
\item Suppose the initial speed is 50 km/h, how high will the object get?
Compute the speed at the moment of maximum height.
\item At what time will the object fall to the ground?
With what speed?
\end{enumerate}
}
%SOLUTION
{\begin{enumerate}
\item Speed $v(t)=v_0-gt$ and acceleration $a(t)=-g$.
\item Maximum height $9.83$ m at $1.42$ s. The speed at that moment vanishes.
\item The object fall to the ground at $2.83$ s with speed $-13.89$ m/s.
\end{enumerate}
}
%RESOLUTION
{
}


\newproblem{der-27}{qui}{}
%STATEMENT
{A liquid solution is kept in a cylindrical pipette of radius 5 mm.
Suppose the liquid is taken out of the pipette at a rate of 0.5 ml per second;
compute the rate of change of the level of liquid in the pipette.
}
%SOLUTION
{$-0.00637$ cm/s.
}
%RESOLUTION
{
}


\newproblem{der-28}{qui}{}
%STATEMENT
{Radioactive decay is given by the following function:
\[
m(t) = m_0e^{-kt},
\]
where $m(t)$ denotes the amount of matter at time $t$, $m_0$ is the initial amount of radioactive matter, and $k$ is a constant called the
\emph{decay constant}.
The variable $t$ represents time.
Compute the speed of decay at any given time $t$.\\
Recall that the \emph{half life} of a radioactive material is the time it takes for a quantity to reduce to half its initial value.
Suppose for certain radioactive material we have $k=0.002$, compute the half life of the material.
}
%SOLUTION
{Speed of decay: $-km_0e^{-kt}$.\\
Half life: 346.57 years.
}
%RESOLUTION
{
}


\newproblem{der-29}{med}{}
%STATEMENT
{The radius of a sphericall cell is equal to 5 $\mu$m, with a possible error of $0.2$ $\mu$m; compute the error in the measurement of the
area of the cell.
More generally, if the error in the measurment of the radius is $2\%$, what is the error in the value of the surface of
the cell?\\
Remark: The surface of a sphere of radius $r$ is given by $S=4\pi r^2$.
Solve the problem by means of the linear approximation (tangent line) of a function.
}
%SOLUTION
{For an radius error of $0.2$ $\mu$m the approximate error in the area is $8\pi$ $\mu\mbox{m}^2$, and for a relative error of $2\%$
the approximate relative error in the area is $4\%$.}
%RESOLUTION
{
}


\newproblem{der-30}{med}{}
%STATEMENT
{Blood flows through an artery at a speed $v$, which is related to the radius $r$ of the artery by the following expression, known as
Poiseuille's law,
\[
v(r) = cr^2.
\]
As mentioned above, $v$ is the speed of the flow, $r$ the radius of the artery, which we will assume to be cylindrical, and $c$ is a constant.
Assume the radius can be measured with a precision of 5\%; calculate the precision in the in the computation of the speed.
}
%SOLUTION
{$10\%$.
}
%RESOLUTION
{
}


\newproblem{der-31}{qui}{}
%STATEMENT
{A cylinder of radius $r = 4$ cm and height $h = 3$ is heated, and so its dimensions change with speed given by $\dfrac{dr}{dt}=\dfrac{dh}{dt}= 1$ cm/s.
Find the approximate change in the volume of the cylinder at 5 and 10 seconds after the heating process starts.
}
%SOLUTION
{$dV = 2\pi r h dt + \pi r^2 dt$ and at the initial moment $dV = 40\pi dt$. 5 seconds after the approximate rate of change is $dV(5) = 40\pi 5 = 200\pi$ cm$^3$/s, and 10 seconds after $dV(10) = 40\pi 10 = 400\pi$ cm$^3$/s.
}
%RESOLUTION
{
}


\newproblem{der-32}{amb}{}
%STATEMENT
{The speed $v(n)$ at which a plant grows depends on the amount of nitrogen available $n$ by the following relation:
\[
v(n) = \frac{an}{k+n},	\quad n\geq 0,
\]
where $a$ and $k$ are positive constants.
Study the growth of this function, and explain your results.
}
%SOLUTION
{The speed increases as $n$ increases but each time with less force, so that for $n\rightarrow \infty$ the speed becomes stable. }
%RESOLUTION
{
}


\newproblem{der-33}{qui}{}
%STATEMENT
{The pH measures the concentration of hydrogen ions H$^+$ in an aqueous solution.
It is defined by
\[
\mbox{pH} = -\log_{10}(\mbox{H}^+).
\]
Compute the derivative of the pH as a function of the concentration of H$^+$.
Study the growth of the pH function.
}
%SOLUTION
{The pH decreases as the concentration of hydrogen ions H$^+$ increase.
}
%RESOLUTION
{
}


\newproblem{der-34}{gen}{}
%STATEMENT
{Compute the derivative function of $f(x)=x^3-2x^2+1$ at the points $x=-1$, $x=0$ and $x=1$.
Explain your result.
Find an equation of the tangent line to the graph of $f$ at each of the three given points.
}
%SOLUTION
{$f'(-1)=7$, $f'(0)=0$ y $f'(1)=-1$.\\
Tangent line at $x=-1$: $y=-2+7(x+1)$.\\
Tangent line at $x=0$: $y=1$.\\
Tangent line at $x=1$: $y=-(x-1)$.
}
%RESOLUTION
{
}

\newproblem{der-35}{qui}{*}
%STATEMENT
{
In certain chemical process, the concentration of certain substance $c$ depends on the concentration of two other substances $a$
and $b$, by the following equation $c=\sqrt[3]{ab^2}$.
Suppose that at certain moment, when $a=b=2$ mg/mm$^3$, the concentrations of $a$ and $b$ increase at rates of $0.2$ mg$\cdot$ mm$^{-3}$/s, and
$0.4$ mg$\cdot$ mm$^{-3}$/s, respectively.
Approximate the concentration of $c$ after 2 seconds.
}
%SOLUTION
{$c'(t_0)=1/3$ mg$\cdot$mm$^{-3}$/s.\\
$c(t_0+2)\approx 8/3$  mg$\cdot$mm$^{-3}$.
}
%RESOLUTION
{
}
